
\section{Atiyah-Segal Completion Theorem}

\begin{definition}[Representation Ring]
Let $\{V_1, V_2, \dots, V_k\}$ be the complete set of non-isomorphic irreducible complex representations of a finite group $G$.
The representation ring $R(G)$ is defined as the set of all formal linear combinations of these irreducible representations with \textbf{integer coefficients}:
$$ R(G) = \left\{ \sum_{i=1}^k n_i V_i \mid n_i \in \mathbb{Z} \right\} $$

The ring structure is defined by:
\begin{enumerate}
    \item \textbf{Addition:} Component-wise addition of integers.
    $$ \left(\sum n_i V_i\right) + \left(\sum m_i V_i\right) = \sum (n_i + m_i) V_i $$
    \item \textbf{Multiplication:} Defined by the tensor product, extended linearly. If $V_i \otimes V_j = \bigoplus_l c_{ij}^l V_l$, then:
    $$ V_i \cdot V_j = \sum_{l=1}^k c_{ij}^l V_l $$
\end{enumerate}
Elements with negative coefficients (where some $n_i < 0$) are called \textbf{virtual representations}.
\end{definition}

\begin{definition}[Augmentation Ideal $I(G)$]
The augmentation ideal $I(G)$ is the kernel of the dimension homomorphism $\varepsilon: R(G) \to \mathbb{Z}$. It consists of all virtual representations of dimension zero:
$$ I(G) = \{ x \in R(G) \mid \dim(x) = 0 \} $$
It is generated by elements of the form $[V] - \dim(V) \cdot 1$, where $1$ denotes the trivial representation.
\end{definition}

\begin{construction}[$EG$ by Excising Fixed Points]
Let $\mathcal{V} \cong \mathbb{R}^\infty$ be an infinite-dimensional vector space equipped with a faithful linear action of the finite group $G$ (e.g., the infinite direct sum of the regular representation).

We define the \textbf{singular set} $Z$ as the union of the fixed-point subspaces of all non-identity elements:
$$ Z := \bigcup_{g \in G, g \neq e} \mathcal{V}^g = \{ v \in \mathcal{V} \mid \exists g \neq e, g \cdot v = v \} $$

The universal bundle $EG$ is constructed as the complement of this set:
$$ EG := \mathcal{V} \setminus Z $$

The classifying space is the quotient $BG = EG/G$.
\end{construction}

\begin{construction}[The Classifying Space $BG$]
Given the universal space $EG = \mathcal{V} \setminus Z$ constructed above, where the action of $G$ is free.
We define the \textbf{classifying space} $BG$ as the orbit space of this action:
$$ BG := EG / G = \{ G \cdot v \mid v \in EG \} $$
\end{construction}


\begin{construction}[The Natural Map $\alpha: R(G) \to K(BG)$]
Let $G \to EG \to BG$ be the universal principal bundle constructed previously. We define the map $\alpha$ by converting algebraic representations into geometric vector bundles via the \textbf{associated bundle construction}.

Given a finite-dimensional complex representation $V$ of $G$:

\begin{enumerate}
    \item \textbf{The Mixing (Balanced Product):}
    Consider the product space $EG \times V$. We define the associated vector bundle $\mathcal{E}_V$ over $BG$ as the quotient by the diagonal $G$-action:
    $$ \mathcal{E}_V := EG \times_G V = (EG \times V) / \sim $$
    where the equivalence relation is defined by $(g \cdot e, g \cdot v) \sim (e, v)$ for all $g \in G$.

    \item \textbf{The Mapping:}
    The map $\alpha$ assigns to each representation class $[V]$ the homotopy class of its associated bundle $[\mathcal{E}_V]$:
    $$ \alpha: R(G) \longrightarrow K(BG) $$
    $$ [V] \longmapsto [EG \times_G V] $$
\end{enumerate}

This map $\alpha$ is a ring homomorphism, translating the algebraic data of $G$-representations into the topological data of flat vector bundles over $BG$.
\end{construction}

\begin{theorem}[Atiyah-Segal Completion Theorem]
Let $G$ be a finite group. The natural map $\alpha: R(G) \to K(BG)$ induces an isomorphism of rings between the $I$-adic completion of the representation ring and the topological K-theory of the classifying space:

$$ \widehat{R(G)}_I \xrightarrow{\; \cong \;} K(BG) $$

where $I \subset R(G)$ is the augmentation ideal and $\widehat{R(G)}_I = \varprojlim_{n} R(G)/I^n$.
\end{theorem}


\begin{remark}[Flatness and Torsion]
Since a finite group $G$ has the discrete topology, the map $EG \to BG$ is a covering map. Consequently, every associated bundle $\mathcal{E}_V$ is canonically \textbf{flat} (possessing zero curvature).

This implies that all \textbf{rational} Chern classes vanish via Chern-Weil theory:
$$ c_k(\mathcal{E}_V) \otimes \mathbb{Q} = 0 $$
Therefore, unlike the case of Lie groups, the Atiyah-Segal theorem for finite groups describes an isomorphism that is purely about the \textbf{torsion} information in the K-theory of $BG$.
\end{remark}
