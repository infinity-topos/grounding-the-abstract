\lecture{Lie Algebras and Homotopy Theory}{Based on Notes}

\begingroup
% --- [Local Symbol Definitions] ---
\newcommand{\OmX}{\Omega X}
\newcommand{\piQ}{\pi_{*}(X) \otimes \mathbb{Q}}
\newcommand{\LieAlg}{\mathcal{L}ie}
\newcommand{\catC}{\mathcal{C}}
\newcommand{\catA}{\mathcal{A}}
\newcommand{\catE}{\mathcal{E}}

% --- [Main Content] ---

\section{Group Structures and Loop Spaces}

Recall that if $G$ is a group, we have the commutator map $G \times G \to G$ defined by $(x,y) \mapsto xyx^{-1}y^{-1}$. If $G$ is a Lie group, differentiation of the commutator map yields the Lie bracket on the tangent space at the identity:
\[
    [\cdot, \cdot] : T_e G \times T_e G \to T_e G.
\]

We wish to apply this intuition to topological spaces. Let $X$ be a topological space with a base point $x_0 \in X$.
\begin{definition}[Loop Space]
    The loop space $\OmX$ is defined as the space of based paths:
    \[
        \OmX = \{ p : [0,1] \to X \mid p(0) = p(1) = x_0 \}.
    \]
\end{definition}

There is a well-known isomorphism relating the fundamental group of the loop space to the homotopy groups of the base space:
\[
    \pi_1(\OmX) \cong \pi_2(X), \quad \text{and generally} \quad \pi_k(\OmX) \cong \pi_{k+1}(X).
\]

The space $\OmX$ admits a binary operation via concatenation of loops. Let $p, q \in \OmX$. We define the product $p * q$ as:
\[
    (p * q)(t) = 
    \begin{cases} 
        p(2t) & 0 \le t \le \frac{1}{2} \\
        q(2t-1) & \frac{1}{2} \le t \le 1 
    \end{cases}
\]
This operation is not strictly associative, but it is \textit{associative up to homotopy}. Similarly, inversion is defined up to homotopy by $\bar{p}(t) = p(1-t)$.

We can define a map $\OmX \times \OmX \to \OmX$ analogous to the group commutator:
\[
    (p, q) \mapsto (p * q) * (\bar{p} * \bar{q}).
\]
This construction leads to algebraic structures on homotopy groups similar to Lie algebras.

\section{The Whitehead Bracket and Graded Lie Algebras}

The commutator in the loop space induces the \textbf{Whitehead bracket}.
\begin{definition}[Whitehead Bracket]
    The Whitehead bracket is a bilinear map:
    \[
        [\cdot, \cdot] : \pi_{a+1}(X) \times \pi_{b+1}(X) \to \pi_{a+b+1}(X).
    \]
\end{definition}

We can organize the homotopy groups into a graded object. Let us define the shifted graded vector space $L_*$ by:
\[
    L_n := \pi_{n+1}(X).
\]
Then the total space $L = \bigoplus_{n \ge 0} L_n = \bigoplus_{n \ge 0} \pi_{n+1}(X)$ carries the structure of a \textbf{Graded Lie Algebra (GLA)}.

\begin{theorem}[Structure of Homotopy Groups]
    The bracket defined above satisfies:
    \begin{enumerate}
        \item \textbf{Graded Skew-symmetry:} For $x \in L_a, y \in L_b$,
        \[
            [x, y] + (-1)^{|x||y|} [y, x] = 0.
        \]
        \item \textbf{Graded Jacobi Identity:}
        \[
            [x, [y, z]] = [[x, y], z] + (-1)^{|x||y|} [y, [x, z]].
        \]
    \end{enumerate}
\end{theorem}

\begin{remark}[Homotopy Operations]
    A homotopy operation of $n$-variables is a map:
    \[
        \pi_{a_1}(X) \times \dots \times \pi_{a_n}(X) \to \pi_{b}(X).
    \]
    The Whitehead bracket is a 2-variable operation. The \textbf{Hilton-Milnor Theorem} essentially states that all such homotopy operations can be built from 1-variable operations and the Whitehead bracket.
\end{remark}

\section{Rational Homotopy Theory}

We now transition to \textbf{Rational Homotopy Theory}, where we ignore torsion by tensoring with $\mathbb{Q}$.
\[
    \pi_*(X) \otimes \mathbb{Q}.
\]
This simplifies calculations significantly (Type II algebraic topology—computable and concrete).

\subsection{Quillen's Theorem}
Quillen established an equivalence between the homotopy category of simply connected rational spaces and the category of Differential Graded Lie Algebras (DGLA).

\begin{definition}[Differential Graded Lie Algebra]
    A DGLA is a pair $(L_*, d)$ where $L_*$ is a graded Lie algebra and $d: L_* \to L_{*-1}$ is a differential satisfying:
    \begin{enumerate}
        \item $d^2 = 0$.
        \item \textbf{Leibniz Rule:} $d[x, y] = [dx, y] + (-1)^{|x|} [x, dy]$.
    \end{enumerate}
\end{definition}

Taking the homology of a DGLA, $H_*(L_*, d)$, yields a graded Lie algebra. In Quillen's model for a space $X$, we have:
\[
    H_*(L_*(X)) \cong \pi_{*+1}(X) \otimes \mathbb{Q}.
\]

\begin{definition}[Rational Homotopy Equivalence]
    A map $f: X \to Y$ between simply connected spaces is a \textit{rational homotopy equivalence} if it induces an isomorphism on rational homotopy groups:
    \[
        f_*: \pi_*(X) \otimes \mathbb{Q} \xrightarrow{\cong} \pi_*(Y) \otimes \mathbb{Q}.
    \]
    In the DGLA setting, this corresponds to a \textbf{quasi-isomorphism} (a map inducing an isomorphism on homology) $L_*(X) \to L_*(Y)$.
\end{definition}

\begin{theorem}[Quillen]
    The construction $X \mapsto L_*(X)$ defines an equivalence of categories:
    \[
        \left\{ \begin{matrix} \text{Simply connected} \\ \text{rational homotopy types} \end{matrix} \right\}
        \longleftrightarrow
        \left\{ \begin{matrix} \text{Reduced DGLAs over } \mathbb{Q} \\ \text{up to quasi-isomorphism} \end{matrix} \right\}.
    \]
\end{theorem}

\section{Lie Algebras in General Categories}

We can generalize the notion of a Lie algebra to an arbitrary category $\catA$, provided $\catA$ has sufficient structure.

\begin{definition}[Requirements for $\catA$]
    Let $\catA$ be a category that is:
    \begin{enumerate}
        \item Cocomplete (has all colimits).
        \item Additive.
        \item Symmetric Monoidal: There exists a tensor product $\otimes: \catA \times \catA \to \catA$ that is commutative and associative up to isomorphism, preserving colimits in each variable.
    \end{enumerate}
\end{definition}

A \textbf{Lie algebra object} $L \in \catA$ is equipped with a bracket morphism $br: L \otimes L \to L$ satisfying skew-symmetry and the Jacobi identity.

\begin{example}
    \begin{itemize}
        \item If $\catA = \mathbf{Ab}$ (Abelian groups), $\LieAlg(\catA)$ is the category of standard Lie algebras.
        \item If $\catA = \mathbf{GrAb}$ (Graded Abelian groups), $\LieAlg(\catA)$ is the category of Graded Lie Algebras.
        \item If $\catA = \mathbf{Ch}$ (Chain complexes), $\LieAlg(\catA)$ is the category of DGLAs.
    \end{itemize}
\end{example}

\section{The Universal Category for Lie Algebras}

We can formalize the category of Lie algebras using a universal construction.

\begin{claim}
    There exists a \textbf{universal category} $\catE$ equipped with a generic Lie algebra object $L_u$.
\end{claim}

This allows us to define the category of Lie algebras in any symmetric monoidal category $\catA$ via functors.

\begin{definition}[Functorial Definition of Lie Algebras]
    The category of Lie algebras in $\catA$, denoted as $\LieAlg(\catA)$, is defined as the category of tensor functors from $\catE$ to $\catA$ that preserve colimits:
    \[
        \LieAlg(\catA) \triangleq \left\{ F: \catE \to \catA \mid F \text{ is a } \otimes\text{-functor preserving colimits} \right\}.
    \]
    Under this correspondence, the value of the functor on the universal object recovers the Lie algebra: $F(L_u) \in \LieAlg(\catA)$.
\end{definition}

\subsection{Structure of the Universal Category}
The universal category $\catE$ is constructed such that its objects are generated by tensor powers of the universal Lie algebra object $L_u$.
\begin{itemize}
    \item Objects in $\catE$ are of the form of colimits of direct sums:
    \[
        \bigoplus_{\alpha} L_u^{\otimes n_\alpha} \to \bigoplus_{\beta} L_u^{\otimes m_\beta}.
    \]
    \item $\catE$ serves as an abelian category containing $L_u$.
\end{itemize}

\begin{remark}[Homological Context]
    Recall that if $\catA$ is an abelian category, the category of chain complexes $\mathbf{Ch}(\catA)$ admits a projective generator. We can consider the derived category:
    \[
        D(\catA) = \mathbf{Ch}(\catA)[\text{quasi-iso}^{-1}].
    \]
    There is a construction of the Lie functor in the derived setting:
    \[
        \LieAlg: D(\catE) \to D(\mathbf{Ab}).
    \]
\end{remark}

\subsection{Connection to Finite Sets (Operads)}
The structure of this universal category is closely related to the combinatorics of finite sets.

\begin{definition}
    Let $\text{Fin}^{\text{surj}}$ be the category of finite sets with surjective maps.
\end{definition}

The tensor product of functors $F, G: \text{Fin}^{\text{surj}} \to \catA$ is given by the convolution formula:
\[
    (F \otimes G)(I) = \bigoplus_{J \subseteq I} F(J) \otimes G(I \setminus J).
\]
This formalism leads to the duality in operad theory (Koszul Duality), where we have an equivalence of derived categories:
\[
    D(\catE) \cong D(\catE').
\]
\endgroup