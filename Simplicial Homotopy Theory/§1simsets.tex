\documentclass[10pt, a4paper, twocolumn]{article}
\usepackage[margin=0.5in]{geometry}
\usepackage{amsmath, amssymb, amsthm}
\usepackage{xcolor}
\usepackage{enumitem}
\usepackage{titlesec}
\usepackage{tikz-cd}

% Spacing Settings
\setlength{\parindent}{0pt}
\setlist{nosep}
\titlespacing*{\section}{0pt}{5pt}{2pt}
\titlespacing*{\subsection}{0pt}{3pt}{1pt}

\begin{document}

\section*{§ 1. Basic Definitions}

\subsection*{1. Simplicial Sets}
Let $\Delta$ be the category of finite ordinal numbers.
\begin{itemize}
    \item \textbf{Objects:} Ordered sets $\mathbf{n} = \{0 \to 1 \to \dots \to n\}$ for $n \ge 0$.
    \item \textbf{Morphisms:} Order-preserving maps $\theta: \mathbf{m} \to \mathbf{n}$.
\end{itemize}

A \textbf{simplicial set} is a contravariant functor:
$$ X: \Delta^{op} \to \textbf{Sets} $$
It consists of sets $X_n$ (n-simplices) and structure maps induced by morphisms in $\Delta$.

\subsection*{2. Coface and Codegeneracies}
Special morphisms in $\Delta$ generate all morphisms:
\begin{itemize}
    \item \textbf{Coface maps} $d^i: \mathbf{n-1} \to \mathbf{n}$ ($0 \le i \le n$): The injective map missing $i$.
    \item \textbf{Codegeneracy maps} $s^j: \mathbf{n+1} \to \mathbf{n}$ ($0 \le j \le n$): The surjective map covering $j$ twice.
\end{itemize}

\textbf{Cosimplicial Identities (in $\Delta$):}
\begin{align*}
    d^j d^i &= d^i d^{j-1} & \textcolor{red}{\text{if } i < j} \\
    s^j d^i &= d^i s^{j-1} & \textcolor{red}{\text{if } i < j} \\
    s^j d^j &= \text{id} = s^j d^{j+1} & \\
    s^j d^i &= d^{i-1} s^j & \textcolor{red}{\text{if } i > j+1} \\
    s^j s^i &= s^i s^{j+1} & \textcolor{red}{\text{if } i \le j}
\end{align*}

\textbf{Simplicial Identities (Structure maps of $X$):}
For a simplicial set $X$, let $d_i = X(d^i)$ (faces) and $s_j = X(s^j)$ (degeneracies):
\begin{align*}
    d_i d_j &= d_{j-1} d_i & \textcolor{red}{\text{if } i < j} \\
    d_i s_j &= s_{j-1} d_i & \textcolor{red}{\text{if } i < j} \\
    d_j s_j &= \text{id} = d_{j+1} s_j & \\
    d_i s_j &= s_j d_{i-1} & \textcolor{red}{\text{if } i > j+1} \\
    s_i s_j &= s_{j+1} s_i & \textcolor{red}{\text{if } i \le j}
\end{align*}

\subsection*{3. Simplicial Abelian Groups}
A \textbf{simplicial abelian group} is a functor:
$$ A: \Delta^{op} \to \textbf{Ab} $$
where $\mathbf{Ab}$ is the category of abelian groups.
\begin{itemize}
    \item Let $\mathbb{Z}Y$ be the free abelian group on a simplicial set $Y$.
    \item \textbf{Moore Complex:} A chain complex defined by:
    $$ (\mathbb{Z}Y)_n \xrightarrow{\partial} (\mathbb{Z}Y)_{n-1} $$
    with boundary $\partial = \sum_{i=0}^n (-1)^i d_i$.
\end{itemize}

\subsection*{4. Classifying Space (Nerve)}
Let $\mathcal{C}$ be a small category. The \textbf{classifying space} (or nerve) $B\mathcal{C}$ is a simplicial set defined by:
$$ B\mathcal{C}_n = \text{hom}_{\textbf{cat}}(\mathbf{n}, \mathcal{C}) $$
An $n$-simplex is a string of composable arrows of length $n$:
$$ a_0 \to a_1 \to \dots \to a_n $$

\subsection*{5. Standard n-Simplex \& Yoneda Lemma}
The \textbf{standard n-simplex} $\Delta^n$ is the representable functor:
$$ \Delta^n = \text{hom}_{\Delta}(-, \mathbf{n}) $$
\textbf{Yoneda Lemma:} There is a natural bijection between $n$-simplices of $X$ and simplicial maps from $\Delta^n$ to $X$:
$$ \text{hom}_S(\Delta^n, X) \cong X_n $$
Let $\iota_n = \text{id}_{\mathbf{n}} \in (\Delta^n)_n$. A map $\phi: \Delta^n \to X$ corresponds to the simplex $x = \phi(\iota_n)$. Conversely, $x \in X_n$ defines $\iota_x: \Delta^n \to X$.

\subsection*{6. Boundary ($\partial \Delta^n$) and Horns ($\Lambda^n_k$)}
\textbf{Boundary:} $\partial \Delta^n \subset \Delta^n$ is the smallest subcomplex containing all faces $d_j(\iota_n)$.

\textbf{Construction of $\partial \Delta^n$:}
The set of $j$-simplices $(\partial \Delta^n)_j$ is defined as:
$$
(\partial \Delta^n)_j = 
\begin{cases} 
(\Delta^n)_j & \textcolor{red}{\text{if } 0 \le j \le n-1} \\
\text{deg}_j & \textcolor{red}{\text{if } j \ge n}
\end{cases}
$$
where $\text{deg}_j$ denotes the set of iterated degeneracies of elements in $(\Delta^n)_{n-1}$. Essentially, in dimensions $\ge n$, it contains only degenerate elements originating from lower dimensions.

\textbf{Example:} $\partial \Delta^3$ (Boundary of the 3-simplex).
\begin{itemize}
    \item It contains all proper faces of $\Delta^3$ but not the non-degenerate 3-simplex $\iota_3$ (or its degeneracies).
    \item \textbf{Non-degenerate elements:}
    \begin{itemize}
        \item 4 vertices (0-simplices).
        \item 6 edges (1-simplices).
        \item 4 faces (2-simplices): $d_0(\iota_3), d_1(\iota_3), d_2(\iota_3), d_3(\iota_3)$.
    \end{itemize}
    \item Geometrically, it forms the surface of a tetrahedron.
\end{itemize}

\textbf{k-th Horn:} $\Lambda^n_k \subset \Delta^n$ ($n \ge 1, 0 \le k \le n$) is the subcomplex generated by all faces $d_j(\iota_n)$ \textbf{except} the $k$-th face $d_k(\iota_n)$.

\textbf{Example:} $\Lambda^3_2$ (The 2nd horn of the 3-simplex) is the subcomplex of $\Delta^3$ generated by the faces $d_0, d_1,$ and $d_3$ (omitting $d_2$).

\end{document}