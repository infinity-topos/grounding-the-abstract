\documentclass[10pt, a4paper, twocolumn]{article}
\usepackage[margin=0.5in]{geometry}
\usepackage{amsmath, amssymb, amsthm}
\usepackage{xcolor}
\usepackage{enumitem}
\usepackage{titlesec}
\usepackage{tikz-cd}
\usepackage{mathrsfs}

% 定义 section 的样式
\titleformat{\section}
  {\large\bfseries}      % 字体格式 (粗体, 稍大)
  {\S\,\thesection.}     % 标签格式:§ 符号 + 微小空格 + 编号 + 点
  {0.5em}                % 编号与标题之间的间距
  {}

\titleformat{\subsection}
  {\bfseries}            % 字体格式
  {\thesubsection.}      % 标签格式:编号 + 点
  {0.5em}
  {}

% Spacing Settings
\setlength{\parindent}{0pt}
\setlist{nosep}
\titlespacing*{\section}{0pt}{5pt}{2pt}
\titlespacing*{\subsection}{0pt}{3pt}{1pt}

% Macros
\newcommand{\cat}[1]{\mathbf{#1}}
\newcommand{\Top}{\cat{Top}}
\newcommand{\SSet}{\cat{S}}
\newcommand{\CGHaus}{\cat{CGHaus}}
\newcommand{\Hom}{\mathrm{Hom}}
\newcommand{\colim}{\operatorname*{colim}}


\begin{document}

\section{Simplicial Sets}

\subsection{1. Simplicial Sets}
Let $\Delta$ be the category of finite ordinal numbers.
\begin{itemize}
    \item \textbf{Objects:} Ordered sets $\mathbf{n} = \{0 \to 1 \to \dots \to n\}$ for $n \ge 0$.
    \item \textbf{Morphisms:} Order-preserving maps $\theta: \mathbf{m} \to \mathbf{n}$.
\end{itemize}

A \textbf{simplicial set} is a contravariant functor:
$$ X: \Delta^{op} \to \textbf{Sets} $$
It consists of sets $X_n$ (n-simplices) and structure maps induced by morphisms in $\Delta$.

\subsection*{2. Coface and Codegeneracies}
Special morphisms in $\Delta$ generate all morphisms:
\begin{itemize}
    \item \textbf{Coface maps} $d^i: \mathbf{n-1} \to \mathbf{n}$ ($0 \le i \le n$): The injective map missing $i$.
    \item \textbf{Codegeneracy maps} $s^j: \mathbf{n+1} \to \mathbf{n}$ ($0 \le j \le n$): The surjective map covering $j$ twice.
\end{itemize}

\textbf{Cosimplicial Identities (in $\Delta$):}
\begin{align*}
    d^j d^i &= d^i d^{j-1} & \textcolor{red}{\text{if } i < j} \\
    s^j d^i &= d^i s^{j-1} & \textcolor{red}{\text{if } i < j} \\
    s^j d^j &= \text{id} = s^j d^{j+1} & \\
    s^j d^i &= d^{i-1} s^j & \textcolor{red}{\text{if } i > j+1} \\
    s^j s^i &= s^i s^{j+1} & \textcolor{red}{\text{if } i \le j}
\end{align*}

\textbf{Simplicial Identities (Structure maps of $X$):}
For a simplicial set $X$, let $d_i = X(d^i)$ (faces) and $s_j = X(s^j)$ (degeneracies):
\begin{align*}
    d_i d_j &= d_{j-1} d_i & \textcolor{red}{\text{if } i < j} \\
    d_i s_j &= s_{j-1} d_i & \textcolor{red}{\text{if } i < j} \\
    d_j s_j &= \text{id} = d_{j+1} s_j & \\
    d_i s_j &= s_j d_{i-1} & \textcolor{red}{\text{if } i > j+1} \\
    s_i s_j &= s_{j+1} s_i & \textcolor{red}{\text{if } i \le j}
\end{align*}

\subsection*{3. Simplicial Abelian Groups}
A \textbf{simplicial abelian group} is a functor:
$$ A: \Delta^{op} \to \textbf{Ab} $$
where $\mathbf{Ab}$ is the category of abelian groups.
\begin{itemize}
    \item Let $\mathbb{Z}Y$ be the free abelian group on a simplicial set $Y$.
    \item \textbf{Moore Complex:} A chain complex defined by:
    $$ (\mathbb{Z}Y)_n \xrightarrow{\partial} (\mathbb{Z}Y)_{n-1} $$
    with boundary $\partial = \sum_{i=0}^n (-1)^i d_i$.
\end{itemize}

\subsection*{4. Classifying Space (Nerve)}
Let $\mathcal{C}$ be a small category. The \textbf{classifying space} (or nerve) $B\mathcal{C}$ is a simplicial set defined by:
$$ B\mathcal{C}_n = \text{hom}_{\textbf{cat}}(\mathbf{n}, \mathcal{C}) $$
An $n$-simplex is a string of composable arrows of length $n$:
$$ a_0 \to a_1 \to \dots \to a_n $$

\subsection*{5. Standard n-Simplex \& Yoneda Lemma}
The \textbf{standard n-simplex} $\Delta^n$ is the representable functor:
$$ \Delta^n = \text{hom}_{\Delta}(-, \mathbf{n}) $$
\textbf{Yoneda Lemma:} There is a natural bijection between $n$-simplices of $X$ and simplicial maps from $\Delta^n$ to $X$:
$$ \text{hom}_S(\Delta^n, X) \cong X_n $$
Let $\iota_n = \text{id}_{\mathbf{n}} \in (\Delta^n)_n$. A map $\phi: \Delta^n \to X$ corresponds to the simplex $x = \phi(\iota_n)$. Conversely, $x \in X_n$ defines $\iota_x: \Delta^n \to X$.

\subsection*{6. Boundary ($\partial \Delta^n$) and Horns ($\Lambda^n_k$)}
\textbf{Boundary:} $\partial \Delta^n \subset \Delta^n$ is the smallest subcomplex containing all faces $d_j(\iota_n)$.

\textbf{Construction of $\partial \Delta^n$:}
The set of $j$-simplices $(\partial \Delta^n)_j$ is defined as:
$$
(\partial \Delta^n)_j = 
\begin{cases} 
(\Delta^n)_j & \textcolor{red}{\text{if } 0 \le j \le n-1} \\
\text{deg}_j & \textcolor{red}{\text{if } j \ge n}
\end{cases}
$$
where $\text{deg}_j$ denotes the set of iterated degeneracies of elements in $(\Delta^n)_{n-1}$. Essentially, in dimensions $\ge n$, it contains only degenerate elements originating from lower dimensions.

\textbf{Example:} $\partial \Delta^3$ (Boundary of the 3-simplex).
\begin{itemize}
    \item It contains all proper faces of $\Delta^3$ but not the non-degenerate 3-simplex $\iota_3$ (or its degeneracies).
    \item \textbf{Non-degenerate elements:}
    \begin{itemize}
        \item 4 vertices (0-simplices).
        \item 6 edges (1-simplices).
        \item 4 faces (2-simplices): $d_0(\iota_3), d_1(\iota_3), d_2(\iota_3), d_3(\iota_3)$.
    \end{itemize}
    \item Geometrically, it forms the surface of a tetrahedron.
\end{itemize}

\textbf{k-th Horn:} $\Lambda^n_k \subset \Delta^n$ ($n \ge 1, 0 \le k \le n$) is the subcomplex generated by all faces $d_j(\iota_n)$ \textbf{except} the $k$-th face $d_k(\iota_n)$.

\textbf{Example:} $\Lambda^3_2$ (The 2nd horn of the 3-simplex) is the subcomplex of $\Delta^3$ generated by the faces $d_0, d_1,$ and $d_3$ (omitting $d_2$).

\section*{§ 1. The Simplex Category}

\subsection*{1. Definition}
Let $X$ be a simplicial set. The \textbf{simplex category} of $X$, denoted $\Delta \downarrow X$, is defined as follows:
\begin{itemize}
    \item \textbf{Objects:} The simplices of $X$, represented as simplicial maps $\sigma: \Delta^n \to X$.
    \item \textbf{Morphisms:} A morphism from $(\sigma: \Delta^n \to X)$ to $(\tau: \Delta^m \to X)$ is a commutative diagram in $\SSet$:
    \[
    \begin{tikzcd}
        \Delta^n \arrow[rr, "\theta^*"] \arrow[dr, "\sigma"'] & & \Delta^m \arrow[dl, "\tau"] \\
        & X &
    \end{tikzcd}
    \]
    where $\theta^*: \Delta^n \to \Delta^m$ is induced by an ordinal map $\theta: m \to n$ in $\Delta$.
\end{itemize}

\subsection*{2. Decomposition Lemma}
Any simplicial set $X$ is a colimit of its simplices:
\[
X \cong \colim_{(\Delta^n \to X) \in \Delta \downarrow X} \Delta^n
\]

\section*{§ 2. Geometric Realization}

\subsection*{1. Definition}
The \textbf{geometric realization} of a simplicial set $X$, denoted $|X|$, is the topological space defined by the colimit:
\[
|X| = \colim_{(\Delta^n \to X) \in \Delta \downarrow X} |\Delta^n|_{\Top}
\]
where $|\Delta^n|_{\Top} \subset \mathbb{R}^{n+1}$ is the standard topological $n$-simplex $\{(t_0, \dots, t_n) \mid \sum t_i = 1, t_i \ge 0\}$.

\subsection*{2. Adjunction}
The realization functor $|\cdot|: \SSet \to \Top$ is left adjoint to the \textbf{singular functor} $S(\cdot): \Top \to \SSet$.
\begin{itemize}
    \item \textbf{Singular Set:} For $Y \in \Top$, $S(Y)_n = \hom_{\Top}(|\Delta^n|, Y)$.
    \item \textbf{Adjunction Isomorphism:} For $X \in \SSet, Y \in \Top$:
    \[
    \hom_{\Top}(|X|, Y) \cong \hom_{\SSet}(X, S(Y))
    \]
\end{itemize}
\textbf{Properties:}
\begin{itemize}
    \item $|\cdot|$ preserves all colimits (as a left adjoint).
    \item The adjunction is natural in $X$ and $Y$.
\end{itemize}

\section*{§ 3. CW Complex Structure}

\subsection*{1. Skeletons ($sk_n X$)}
$|X|$ is a CW-complex. Its structure is defined via the filtration of $X$ by skeletons.
\begin{itemize}
    \item The \textbf{$n$-skeleton} $sk_n X \subseteq X$ is the subcomplex generated by all simplices of $X$ of degree $\le n$.
    \item $X = \bigcup_{n \ge 0} sk_n X$.
\end{itemize}

\subsection*{2. Non-degenerate Simplices}
Let $NX_n$ be the set of \textbf{non-degenerate} $n$-simplices of $X$:
\[
NX_n = \{ x \in X_n \mid x \notin \bigcup_{i=0}^{n-1} s_i(X_{n-1}) \}
\]

\subsection*{3. Pushout Construction}
$|sk_n X|$ is obtained from $|sk_{n-1} X|$ by attaching $n$-cells corresponding to non-degenerate simplices. There is a pushout diagram in $\SSet$ (and consequently in $\Top$ after realization):
\[
\begin{tikzcd}
    \coprod\limits_{x \in NX_n} \partial \Delta^n \arrow[r] \arrow[d, hook] & sk_{n-1} X \arrow[d] \\
    \coprod\limits_{x \in NX_n} \Delta^n \arrow[r] & sk_n X
\end{tikzcd}
\]
where the maps are induced by the characteristic maps of the simplices in $NX_n$.

\section*{§ 4. Standard Simplex Presentation}

The boundary of the standard simplex, $\partial \Delta^n$, and by extension any realization involving boundaries, is governed by the \textbf{cosimplicial identities}.

\subsection*{1. Coequalizer Presentation}
The realization of the boundary $|\partial \Delta^n|$ is the image of the map induced by face operators. It can be described via the coequalizer:
\[
\coprod_{0 \le i < j \le n} |\Delta^{n-2}| \rightrightarrows \coprod_{k=0}^n |\Delta^{n-1}| \to |\partial \Delta^n|
\]

\subsection*{2. Generating Relations}
The coequalizer identifies faces along their common boundaries using the relation:
\[
\textcolor{red}{\text{For } 0 \le i < j \le n:} \quad d^j d^i = d^i d^{j-1}
\]
This ensures that the $(n-1)$-faces of $|\Delta^n|$ are glued correctly along their $(n-2)$-faces to form the boundary $S^{n-1}$.

\section*{§ 5. Kelley Spaces (CGHaus)}

The category of topological spaces $\Top$ is often inconvenient for homotopy theory regarding products. The realization functor is best viewed as taking values in $\CGHaus$ (Compactly Generated Hausdorff spaces).

\subsection*{1. The Product Problem}
In general $\Top$, the natural map is a continuous bijection but not necessarily a homeomorphism:
\[
|X \times Y| \not\cong |X| \times |Y|_{\Top}
\]

\subsection*{2. The Kelley Product}
Let $\times_{Ke}$ denote the product in the category $\CGHaus$ (the $k$-ification of the standard product topology).
\begin{itemize}
    \item \textbf{Theorem:} The realization functor preserves finite products if the target is considered as $\CGHaus$:
    \[
    |X \times Y| \cong |X| \times_{Ke} |Y|
    \]
    \item \textbf{Consequence:} The functor $|\cdot|: \SSet \to \CGHaus$ preserves all finite limits.
\end{itemize}


\end{document}