\documentclass[a4paper,11pt]{article}
\usepackage[utf8]{inputenc}
\usepackage[T1]{fontenc}
\usepackage{amsmath, amssymb, amsthm}
\usepackage{geometry}
\usepackage{xcolor}
\usepackage{hyperref}

\geometry{left=2.5cm, right=2.5cm, top=2.5cm, bottom=2.5cm}

% 定理环境设置
\newtheorem{theorem}{Theorem}[section]
\newtheorem{proposition}[theorem]{Proposition}
\newtheorem{lemma}[theorem]{Lemma}
\theoremstyle{definition}
\newtheorem{definition}[theorem]{Definition}
\newtheorem{example}[theorem]{Example}
\theoremstyle{remark}
\newtheorem{remark}[theorem]{Remark}

% 自定义命令
\newcommand{\Spec}{\operatorname{Spec}}
\newcommand{\m}{\mathfrak{m}}

\title{\textbf{Notes on the Tensor Product of Local Rings}}

\date{\today}

\begin{document}

\maketitle

\section{The Isomorphism Formula of Localized Tensor Products}

Let $A$ and $B$ be commutative rings over a base ring $k$. Let $S \subset A$ and $T \subset B$ be multiplicative subsets.

\begin{theorem}[Localization Commutes with Tensor Product]
The tensor product of two localized rings is isomorphic to the localization of their tensor product with respect to the product of their multiplicative sets:
\begin{equation}
    (S^{-1}A) \otimes_k (T^{-1}B) \cong (S \otimes T)^{-1} (A \otimes_k B)
\end{equation}
where $S \otimes T = \{ s \otimes t \mid s \in S, t \in T \}$ is the multiplicative set generated by simple tensors of the denominators.
\end{theorem}

\subsection*{Interpretation: The "Separable Denominator" Constraint}
The elements of the ring $(S^{-1}A) \otimes_k (T^{-1}B)$ can be viewed as fractions:
$$ \frac{\alpha}{s \otimes t} $$
where $\alpha \in A \otimes_k B$. The crucial observation is that \textbf{allowable denominators must be separable}. That is, a denominator must be strictly of the form $f(a) \cdot g(b)$. A "mixed" element (e.g., $1 - (x+y)$) is generally \textbf{not} a valid denominator, even if it is non-zero at the focal point.

\section{Why Tensor Products of Local Rings are (Often) Non-Local}

Although the formula in Section 1 is a "localization," the resulting ring is generally \textbf{not a local ring} (i.e., it may have more than one maximal ideal).

\begin{example}[The Standard Counterexample]
Consider the localization of polynomial rings at the origin:
$$ A = k[x]_{(x)}, \quad B = k[y]_{(y)} $$
Their tensor product is:
$$ R = k[x]_{(x)} \otimes_k k[y]_{(y)} \cong S^{-1} (k[x,y]) $$
where $S = \{ f(x)g(y) \mid f(0)\neq 0, g(0)\neq 0 \}$.

\paragraph{Proof that $R$ is not local:}
Consider the element $z = 1 - (x+y) \in R$.
\begin{enumerate}
    \item At the origin $(0,0)$, $z$ evaluates to $1 \neq 0$. In the full local ring $k[x,y]_{(x,y)}$, $z$ would be a unit.
    \item However, in $R$, the inverse $z^{-1}$ exists if and only if $1-(x+y)$ divides some element in $S$ (separable polynomials).
    \item Since $1-(x+y)$ is an irreducible polynomial that mixes variables, it cannot divide any $f(x)g(y)$.
    \item Thus, $z$ is \textbf{not invertible} in $R$.
    \item Since $z \notin (x, y)$ (the ideal generated by the origin), $z$ must belong to some \textbf{other} maximal ideal $\m'$.
\end{enumerate}
Therefore, $R$ has at least two maximal ideals: $\m_0 = (x, y)$ and $\m'$.
\end{example}

\section{The Artinian Case: When Locality is Preserved}

If the rings involved are Artinian, the "nilpotency" property forces the result to be local (assuming trivial residue field extension).

\begin{proposition}
Let $A, B$ be Artinian local $k$-algebras with residue fields isomorphic to $k$ (i.e., $A/\m_A \cong k, B/\m_B \cong k$). Then $A \otimes_k B$ is an Artinian local ring.
\end{proposition}

\begin{proof}[Sketch of Proof]
\begin{enumerate}
    \item Since $A, B$ are Artinian, their maximal ideals $\m_A, \m_B$ consist entirely of \textbf{nilpotent elements}.
    \item Any element in $\m_A \otimes B + A \otimes \m_B$ is a sum of nilpotents, hence nilpotent.
    \item Consider the element $u = 1 - (x+y)$ from the previous example. In the Artinian case, $x$ and $y$ are nilpotent. Thus $x+y$ is nilpotent.
    \item \textbf{Key Algebraic Fact:} If $\eta$ is nilpotent, then $1-\eta$ is a unit (inverse is $\sum \eta^i$).
    \item Therefore, any potential "ghost" element like $1-(x+y)$ becomes invertible and cannot generate a new maximal ideal.
    \item The only non-invertible elements are those contained in $\m_{total} = (\m_A, \m_B)$, making it the unique maximal ideal.
\end{enumerate}
\end{proof}

\section{Geometric Interpretation: Generic Points}

Why does $k[x]_{(x)} \otimes k[y]_{(y)}$ have extra maximal ideals? This can be understood via the spectrum of the ring and \textbf{generic points}.

\begin{itemize}
    \item \textbf{The Setup:} $\Spec(A \otimes B)$ roughly corresponds to the product space. The localization process removes all closed points $(a,b)$ where $a \neq 0$ or $b \neq 0$.
    \item \textbf{The Survivor:} The origin $(0,0)$ survives. This corresponds to the maximal ideal $\m = (x, y)$.
    \item \textbf{The Ghosts:} Consider a curve $C$ passing near but not through the origin, e.g., the line $L: x+y=1$.
    \begin{itemize}
        \item In the full plane $\mathbb{A}^2$, this line is defined by a prime ideal $\mathfrak{p} = (x+y-1)$. It is a \textbf{generic point} of the line.
        \item In the localized ring $R$, all specific \textit{closed points} on this line (like $(1,0), (0,1)$) have been removed (turned into units) because their coordinates are non-zero.
        \item However, the generic point itself (the "soul" of the line) remains.
        \item Since there are no closed points "below" it in the partial order of inclusion (they were all removed), this generic point $\mathfrak{p}$ becomes a \textbf{maximal ideal} in $R$.
    \end{itemize}
\end{itemize}

\end{document}