\lecture{Lie Groups}

\begingroup
\begin{definition}
A \textbf{Lie group} is a set $G$ endowed with the structure of a smooth manifold and a group, such that the group operations are smooth. That is, the multiplication map
\[
\mu : G \times G \to G, \quad (g, h) \mapsto gh
\]
and the inversion map
\[
\iota : G \to G, \quad g \mapsto g^{-1}
\]
are both smooth maps.
\end{definition}
\begin{definition}
Let $G$ be a Lie group and let $g \in G$ be a fixed element.
\begin{itemize}
    \item The \textbf{left translation} by $g$ is the map $L_g : G \to G$ defined by
    \[
    L_g(h) = gh \quad \text{for all } h \in G.
    \]
    \item The \textbf{right translation} by $g$ is the map $R_g : G \to G$ defined by
    \[
    R_g(h) = hg \quad \text{for all } h \in G.
    \]
\end{itemize}
Since the group multiplication is smooth, both $L_g$ and $R_g$ are diffeomorphisms of the manifold $G$.
\end{definition}
\begin{definition}
Let $G$ and $H$ be Lie groups. A map $\phi: G \to H$ is called a \textbf{Lie group homomorphism} if:
\begin{enumerate}
    \item $\phi$ is a group homomorphism, meaning $\phi(gh) = \phi(g)\phi(h)$ for all $g, h \in G$; and
    \item $\phi$ is a smooth map between the manifolds $G$ and $H$.
\end{enumerate}
If $\phi$ is in addition a diffeomorphism, then $\phi$ is called a \textbf{Lie group isomorphism}.
\end{definition}
\begin{theorem}
Let $G$ and $H$ be Lie groups, and let $\phi: G \to H$ be a Lie group homomorphism. Then $\phi$ has \textbf{constant rank}.

Specifically, for any $g \in G$, the rank of the differential $d\phi_g: T_g G \to T_{\phi(g)} H$ is equal to the rank of the differential at the identity, $d\phi_e$.
\end{theorem}
\begin{definition}
Let $G$ be a Lie group. A subset $H \subseteq G$ is called a \textbf{Lie subgroup} if:
\begin{enumerate}
    \item $H$ is a subgroup of $G$ in the algebraic sense;
    \item $H$ is endowed with a topology and a smooth structure that make it an \textbf{immersed submanifold} of $G$ (meaning the inclusion map $\iota: H \hookrightarrow G$ is a smooth immersion); and
    \item $H$ is a Lie group with respect to this smooth structure.
\end{enumerate}
It is important to note that the topology on an immersed submanifold $H$ is not necessarily the subspace topology induced from $G$.
\end{definition}
\begin{theorem}[Closed Subgroup Theorem]
Let $G$ be a Lie group and let $H$ be a subgroup of $G$. If $H$ is a closed subset of $G$, then $H$ is an \textbf{embedded Lie subgroup} of $G$.
\end{theorem}
\begin{definition}
Let $G$ be a Lie group and let $M$ be a smooth manifold. A \textbf{left Lie group action} of $G$ on $M$ is a smooth map
\[
\theta : G \times M \to M, \quad (g, p) \mapsto g \cdot p,
\]
satisfying the following two axioms:
\begin{enumerate}
    \item \textbf{Identity:} $e \cdot p = p$ for all $p \in M$, where $e$ is the identity of $G$.
    \item \textbf{Associativity:} $g \cdot (h \cdot p) = (gh) \cdot p$ for all $g, h \in G$ and $p \in M$.
\end{enumerate}
Since the map $\theta$ is smooth, for every $g \in G$, the map $\theta_g : M \to M$ given by $p \mapsto g \cdot p$ is a diffeomorphism.
\end{definition}


\begin{definition}
Let $M$ and $N$ be smooth manifolds endowed with smooth left actions of a Lie group $G$. Let $\theta^M_g : M \to M$ and $\theta^N_g : N \to N$ denote the diffeomorphisms associated with the action of an element $g \in G$.

A smooth map $F : M \to N$ is called \textbf{$G$-equivariant} if it commutes with the $G$-action. That is, for every $g \in G$, the following diagram commutes:

\begin{center}
\begin{tikzcd}[row sep=large, column sep=large]
M \arrow[r, "F"] \arrow[d, "\theta^M_g"']
& N \arrow[d, "\theta^N_g"] \\
M \arrow[r, "F"]
& N
\end{tikzcd}
\end{center}

In algebraic terms, this condition is written as:
\[
F(g \cdot p) = g \cdot F(p) \quad \text{for all } g \in G \text{ and } p \in M.
\]
\end{definition}
\begin{theorem}[Equivariant Rank Theorem]
Let $M$ and $N$ be smooth manifolds endowed with smooth actions of a Lie group $G$. Let $F : M \to N$ be a smooth $G$-equivariant map.
If the action of $G$ on $M$ is \textbf{transitive}, then $F$ has \textbf{constant rank}.

Consequently, if the action is transitive:
\begin{itemize}
    \item The image $F(M)$ is an \textbf{immersed submanifold} of $N$; and
    \item The fibers $F^{-1}(y)$ (for $y \in F(M)$) are closed \textbf{embedded submanifolds} of $M$.
\end{itemize}
\end{theorem}
\begin{theorem}[Rank Theorem]
Let $M$ and $N$ be smooth manifolds of dimension $m$ and $n$, respectively, and let $F: M \to N$ be a smooth map. Suppose that $F$ has \textbf{constant rank} $k$ in a neighborhood of a point $p \in M$.

Then there exist smooth charts $(U, \varphi)$ centered at $p$ (with coordinates $x^1, \dots, x^m$) and $(V, \psi)$ centered at $F(p)$ (with coordinates $y^1, \dots, y^n$) such that the local coordinate representation of $F$,
\[
\hat{F} = \psi \circ F \circ \varphi^{-1},
\]
takes the following canonical form:
\[
\hat{F}(x^1, \dots, x^m) = (x^1, \dots, x^k, \underbrace{0, \dots, 0}_{n-k}).
\]
\textbf{Consequences:}
\begin{itemize}
    \item The level set $F^{-1}(F(p)) \cap U$ is a smooth submanifold of $M$ of dimension $m-k$ (Kernel).
    \item The image $F(U)$ is an immersed submanifold of $N$ of dimension $k$ (Image).
\end{itemize}
\end{theorem}

\begin{remark}
    \textbf{Computation via Jacobian Matrices.}
    Let $X$ and $Y$ be vector fields on $\mathbb{R}^n$ (or on a local chart $U \subset \mathbb{R}^n$), identified with column vectors of component functions $X = (X^1, \dots, X^n)^T$ and $Y = (Y^1, \dots, Y^n)^T$.
    The Lie bracket $[X, Y]$ can be explicitly computed using the Jacobian matrices $J_X = \left( \frac{\partial X^i}{\partial x^j} \right)$ and $J_Y = \left( \frac{\partial Y^i}{\partial x^j} \right)$ via the formula:
    \[
        [X, Y] = J_Y X - J_X Y.
    \]
\end{remark}


\begin{remark}
    \textbf{Tangent Space as the Kernel of the Jacobian.}
    Let $M \subset \mathbb{R}^n$ be a submanifold defined implicitly by the independent equations $f_1(x) = \cdots = f_k(x) = 0$.
    At any point $p \in M$, the differentials $df_1|_p, \dots, df_k|_p$ form a basis for the \textit{conormal space} $N_p^*M$. These differentials appear explicitly as the rows of the Jacobian matrix $J_f(p)$:
    \[
        J_f(p) = \begin{pmatrix} \nabla f_1(p)^T \\ \vdots \\ \nabla f_k(p)^T \end{pmatrix}.
    \]
    Since a tangent vector $v \in T_p M$ must represent a direction in which the defining functions do not change (to first order), the tangent space is precisely the \textbf{kernel} (or null space) of this Jacobian matrix:
    \[
        T_p M = \ker(dF_p) = \left\{ v \in \mathbb{R}^n \mid J_f(p) \cdot v = 0 \right\}.
    \]
    Geometrically, this means the tangent space consists of all vectors orthogonal to the gradients $\{ \nabla f_1, \dots, \nabla f_k \}$.
\end{remark}

\begin{remark}
    \textbf{Decomposition of Volume Forms under Multiple Constraints.}
    Let $N$ be an ambient manifold with a global volume form $\Omega$, and let $M \subset N$ be a submanifold of codimension $k$ defined by the vanishing of $k$ independent functions, $f_1 = \dots = f_k = 0$.
    
    The relationship between the ambient geometry and the submanifold geometry is established by the unique decomposition:
    \[
        \Omega = df_1 \wedge df_2 \wedge \dots \wedge df_k \wedge \omega_M,
    \]
    where:
    \begin{itemize}
        \item The $k$-form $df_1 \wedge \dots \wedge df_k$ represents the \textbf{conormal volume}, generated by the constraint differentials (the rows of the Jacobian) which annihilate the tangent space.
        \item The $(n-k)$-form $\omega_M$ is the \textbf{induced volume form} (or boundary form) on $M$, measuring the volume of the tangent space $TM = \bigcap_{i=1}^k \ker(df_i)$.
    \end{itemize}
    
    In the context of complex geometry or residue theory, this relationship is often denoted using the \textit{Leray residue} notation, intuitively expressing the induced form as a "division" of the ambient form by the defining equations:
    \[
        \omega_M = \frac{\Omega}{df_1 \wedge \dots \wedge df_k} \bigg|_M.
    \]
\end{remark}
\begin{definition}
    \textbf{Orientation Compatibility for Submanifolds.}
    
    Let $X$ be an oriented smooth $n$-manifold with volume form $\Omega_X$. Let $M \subset X$ be a submanifold of dimension $m$ (codimension $k = n-m$).
    
    Suppose the \textbf{normal bundle} $N(M)$ of $M$ is oriented. Let $\nu$ be a non-vanishing $k$-form along $M$ representing this normal orientation (often called the \textit{transverse volume form}).
    
    The \textbf{induced orientation} on $M$ is defined by the unique $m$-form $\omega_M$ satisfying the compatibility condition:
    \[
        \Omega_X \big|_M = \nu \wedge \omega_M.
    \]
    
    In the specific case where $M$ is defined regular level set of independent functions $f_1, \dots, f_k$ (i.e., $M = \{f^{-1}(0)\}$), the normal orientation is naturally given by $\nu = df_1 \wedge \dots \wedge df_k$. The condition becomes:
    \[
        \Omega_X = (df_1 \wedge \dots \wedge df_k) \wedge \omega_M.
    \]
    
    \textit{Note:} This generalizes the boundary case. For a boundary, $k=1$, and $\nu$ corresponds to the single outward normal form. For a general submanifold, $\nu$ represents the "volume" of the directions perpendicular to $M$.
\end{definition}
\begin{definition}
    \textbf{Riemannian Volume Form.}
    
    Let $(M, g)$ be an oriented Riemannian manifold of dimension $n$. The \textbf{Riemannian volume form}, denoted by $dV_g$ (or $\operatorname{vol}_g$), is the unique differential $n$-form on $M$ satisfying the following geometric condition:
    \[
        dV_g(e_1, e_2, \dots, e_n) = 1,
    \]
    for any positively oriented orthonormal basis $\{e_1, \dots, e_n\}$ of the tangent space $T_pM$ at any point $p \in M$.
    
    In terms of any positively oriented local coordinate chart $(U, x^1, \dots, x^n)$, the volume form is given explicitly by:
    \[
        dV_g = \sqrt{\det(g_{ij})} \, dx^1 \wedge dx^2 \wedge \dots \wedge dx^n,
    \]
    where $g_{ij} = g(\frac{\partial}{\partial x^i}, \frac{\partial}{\partial x^j})$ are the components of the metric tensor, and $\sqrt{\det(g_{ij})}$ is the volume density factor.
\end{definition}
\begin{remark}
    \textbf{Converting Scalar Functions to Volume Forms via the Hodge Star.}
    Let $(M, g)$ be an oriented Riemannian $n$-manifold. The mapping from a scalar function $f \in C^\infty(M)$ (a $0$-form) to a top-degree form (an $n$-form) is canonically realized by the \textbf{Hodge star operator} $*$:
    \[
        *: \Omega^0(M) \to \Omega^n(M).
    \]
    
    This isomorphism is defined by multiplication with the Riemannian volume form $dV_g$:
    \[
        *f = f \wedge (*1) = f \cdot dV_g.
    \]
    
    In local coordinates, this transformation encapsulates the density factor:
    \[
        *f = f(x) \sqrt{\det(g_{ij})} \, dx^1 \wedge \dots \wedge dx^n.
    \]
    Consequently, the integral of a function $f$ over $M$ is rigorously defined as the integral of its Hodge dual:
    \[
        \int_M f \coloneqq \int_M *f.
    \]
\end{remark}
\endgroup