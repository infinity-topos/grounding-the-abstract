%----------------------------------------------------------------------------------------
%	SECTION 2: Preliminaries
%----------------------------------------------------------------------------------------

\section{Preliminaries}



{\begingroup


% Categories & Functors
\newcommand{\QCoh}{\operatorname{QCoh}}
\newcommand{\Coh}{\operatorname{Coh}}
\newcommand{\Ind}{\operatorname{Ind}}
\newcommand{\Fun}{\operatorname{Fun}}
\newcommand{\Map}{\operatorname{Map}} % Mapping space
\newcommand{\End}{\operatorname{End}}
\newcommand{\Mod}{\operatorname{Mod}}
\newcommand{\Alg}{\operatorname{Alg}}
\newcommand{\CAlg}{\operatorname{CAlg}}
\newcommand{\Sp}{\operatorname{Sp}} % Spectra
\newcommand{\PSh}{\mathcal{P}}      % Presheaves

% Geometry
\newcommand{\Spec}{\operatorname{Spec}}
\newcommand{\Sch}{\operatorname{Sch}}
\newcommand{\St}{\operatorname{St}} 
\newcommand{\co}{\mathcal{O}} % Structure sheaf

% Limits and Colimits
\newcommand{\limit}{\varprojlim}
\newcommand{\colimit}{\varinjlim}
\newcommand{\tensor}{\otimes}

% Logic & Sets
\newcommand{\infcat}{$\infty$-category}
\newcommand{\infcats}{$\infty$-categories}
\newcommand{\sSet}{\operatorname{Set}_{\Delta}} % Simplicial Sets

\newcommand{\Cat}{\operatorname{Cat}}            % 经典 1-范畴构成的范畴
\newcommand{\CatI}{\mathcal{C}\mathrm{at}_{\infty}} % 无穷范畴构成的无穷范畴
\newcommand{\Kan}{\mathbf{Kan}}                  % Kan 复形全子范畴
\newcommand{\Ob}{\mathrm{Ob}}  

\newcommand{\C}{\mathcal{C}}
\newcommand{\Sfin}{\mathcal{S}}
\newcommand{\Nerve}{N}
\newcommand{\Rig}{\mathfrak{C}}
\newcommand{\Pinf}{\mathcal{P}}
\newcommand{\Psimp}{\mathcal{P}_{\Delta}}
\subsection{$\infty$-categories}


\begin{definition}[$\infty$-Category]
A simplicial set $K$ is an \textbf{$\infty$-category} if for every $n > 1$ and every \textbf{inner} index $0 < i < n$, every map of simplicial sets $f_0: \Lambda^n_i \to K$ admits an extension to an $n$-simplex $f: \Delta^n \to K$.
\[
\begin{tikzcd}
\Lambda^n_i \arrow[d, hook] \arrow[r, "f_0"] & K \\
\Delta^n \arrow[ru, dashed, "f"'] & 
\end{tikzcd}
\]
\end{definition}

\begin{definition}[Simplicial Category]
A \textbf{simplicial category} (or $\sSet$-enriched category) $\mathcal{C}$ is a category where:
\begin{enumerate}
    \item For any two objects $X, Y \in \mathcal{C}$, the collection of morphisms between them is not a set, but a \textbf{simplicial set} $\Map_{\mathcal{C}}(X, Y)$.
    \item For any three objects $X, Y, Z \in \mathcal{C}$, the composition map 
    \[ \Map_{\mathcal{C}}(Y, Z) \times \Map_{\mathcal{C}}(X, Y) \to \Map_{\mathcal{C}}(X, Z) \]
    is a morphism of simplicial sets and satisfies the usual associativity and identity axioms.
\end{enumerate}
A simplicial category $\mathcal{C}$ is \textbf{locally Kan} if for every pair of objects $X, Y \in \Ob(\mathcal{C})$, the mapping simplicial set $\Map_{\mathcal{C}}(X, Y)$ is a Kan complex.
\end{definition}




\begin{definition}[Simplicial Nerve $N_{\Delta}$]
The \textbf{simplicial nerve} $N_{\Delta}(\mathcal{C})$ is the simplicial set defined by the assignment:
\[ 
    N_{\Delta}(\mathcal{C})_n = \mathrm{Hom}_{\Cat_{\Delta}}(\mathfrak{C}[\Delta^n], \mathcal{C}) 
\]
where $\mathfrak{C}[\Delta^n]$ is the \textbf{rigidification} of the $n$-simplex $\Delta^n$ into a simplicial category.
\end{definition}

\begin{definition}[$\infty$-category via $N_\Delta$]
An \textbf{$\infty$-category} (or quasicategory) is a simplicial set $K$ that is equivalent to the simplicial nerve of some locally Kan simplicial category $\mathcal{C}$.
\[ K \simeq N_{\Delta}(\mathcal{C}) \]
\end{definition}





\begin{theorem}[Joyal-Lurie]
There exists a Quillen equivalence between the Joyal model structure on $\sSet$ (modeling quasicategories) and the Bergner model structure on $\Cat_{\Delta}$ (modeling simplicial categories):
\[
\mathfrak{C}[\cdot] : \sSet \rightleftarrows \Cat_{\Delta} : N_{\Delta}.
\]
Specifically, for any simplicial category $\mathcal{C}$ where mapping spaces are Kan complexes, its simplicial nerve $N_{\Delta}(\mathcal{C})$ is a quasicategory.
\end{theorem}




\begin{definition}[Free Cocompletion]
Let $\mathcal{C}$ be a small \infcat{}. An \infcat{} $\PSh(\mathcal{C})$ is called the \textbf{free cocompletion} of $\mathcal{C}$ if it satisfies the following universal property:

\begin{enumerate}
    \item $\PSh(\mathcal{C})$ admits all small colimits.
    \item There exists a functor $j: \mathcal{C} \to \PSh(\mathcal{C})$ (called the Yoneda embedding) such that for any \infcat{} $\mathcal{D}$ which admits small colimits, composition with $j$ induces an equivalence of \infcats{}:
    \[
        \Fun^L(\PSh(\mathcal{C}), \mathcal{D}) \xrightarrow{\sim} \Fun(\mathcal{C}, \mathcal{D}).
    \]
\end{enumerate}
Here, $\Fun^L$ denotes the full subcategory of functors that preserve small colimits (left adjoints).
\end{definition}


\begin{definition}[The \infcat{} of Spaces]
Let $\mathcal{S}$ denote the \infcat{} of spaces. It is defined in two equivalent ways:

\begin{enumerate}
    \item \textbf{Via Dwyer-Kan Localization:} \\
    Let $W$ be the class of weak homotopy equivalences in $\sSet$. We define $\mathcal{S}$ as the homotopy coherent nerve of the simplicial localization:
    \[
        \mathcal{S} \coloneqq N\bigl(\sSet[W^{-1}]\bigr).
    \]
    Equivalently, via Kan complexes: $\mathcal{S} \simeq N(\Kan)$.

    \item \textbf{Via Free Cocompletion:} \\
    The \infcat{} $\mathcal{S}$ is the free cocompletion of the point $*$. That is, it is the category of presheaves:
    \[
        \mathcal{S} \simeq \PSh(*).
    \]
    Universal Property: For any cocomplete \infcat{} $\mathcal{C}$, there is an equivalence $\Fun^L(\mathcal{S}, \mathcal{C}) \simeq \mathcal{C}$.
\end{enumerate}
\end{definition}

\begin{theorem}
Let $K$ be an $\infty$-category (quasi-category). Let $\C = \Rig[K]$ be its associated simplicial category (via rigidification). The construction of the presheaf $\infty$-category commutes with the nerve construction in the following sense:

\begin{enumerate}
    \item \textbf{Simplicial Side:} Consider the category of simplicial presheaves $\Psimp(\C) := \mathrm{Fun}_{\Delta}(\C^{op}, \Sfin_{\text{Kan}})$. This category admits a simplicial model structure (projective structure).
    
    \item \textbf{Infinity Side:} Consider the $\infty$-category of presheaves $\Pinf(K) := \mathrm{Fun}(K^{op}, \Sfin)$.
    
    \item \textbf{Equivalence:} There is an equivalence of $\infty$-categories:
    \[ \Pinf(K) \simeq \Nerve \left( \Psimp(\C)^{\mathrm{cf}} \right) \]
    where $\Psimp(\C)^{\mathrm{cf}}$ denotes the full simplicial subcategory of fibrant-cofibrant objects in the model category of simplicial presheaves.
\end{enumerate}
In summary, the presheaf of an $\infty$-category is modeled by the nerve of the strictly cocomplete simplicial category of enriched presheaves.
\end{theorem}

\begin{remark}[Homotopy Category via Fibrant-Cofibrant Objects]
To correctly construct the homotopy category $\mathrm{Ho}(\mathcal{M})$ from a simplicial model category $\mathcal{M}$, one cannot simply take the path components $\pi_0$ of the mapping spaces between arbitrary objects. 

Instead, one must restrict attention to the full subcategory of \textbf{fibrant-cofibrant objects}, denoted $\mathcal{M}_{cf}$. It is only within this subcategory that the simplicial mapping spaces $\mathrm{Map}_{\mathcal{M}}(X, Y)$ are guaranteed to be Kan complexes representing the correct derived mapping spaces. The morphisms in the homotopy category are thus given by:
\[
[X, Y]_{\mathrm{Ho}(\mathcal{M})} \cong \pi_0 \mathrm{Map}_{\mathcal{M}}(X, Y) \quad \text{for } X, Y \in \mathcal{M}_{cf}.
\]
For general objects $X, Y$, one must first replace them with weakly equivalent fibrant-cofibrant objects (via cofibrant replacement $QX$ and fibrant replacement $RY$) to compute this group.
\end{remark}
\endgroup}

{\begingroup
\newcommand{\C}{\mathcal{C}}
\newcommand{\D}{\mathcal{D}}
\newcommand{\Map}{\mathrm{Map}}
\newcommand{\Sp}{\text{Sp}}
\newcommand{\Exc}{\text{Exc}}
\newcommand{\Sfin}{\mathcal{S}_*^{\text{fin}}}
\newcommand{\uMap}{\underline{\mathrm{Map}}}
\newcommand{\Sone}{\mathbb{S}}
\newcommand{\Vlim}{\varprojlim}
\newcommand{\Clim}{\varinjlim}
\newcommand{\Ssphere}{\mathbb{S}}
\newcommand{\wedgeS}{\wedge}
\newcommand{\Stab}{\mathrm{Stab}} % 稳定化算子
\newcommand{\Sspace}{\mathcal{S}} % 空间范畴
\newcommand{\otimesD}{\otimes_{\mathcal{D}}}
\newcommand{\Unit}{\mathbf{1}_{\mathcal{D}}}

\newcommand{\SigInf}{\Sigma^\infty}    % Suspension Spectrum Functor
\newcommand{\OmeInf}{\Omega^\infty}    % Infinite Loop Space Functor


\subsection{Stable $\infty$-Category}

\begin{definition}[Loop Object]
For any object $X \in \C$, the loop object $\Omega X$ is the limit of the diagram $0 \to X \leftarrow 0$. It fits into the following pullback square:
\[
\begin{tikzcd}
\Omega X \arrow[r] \arrow[d] \arrow[dr, phantom, "\lrcorner", very near start] & 0 \arrow[d] \\
0 \arrow[r] & X
\end{tikzcd}
\]
Intuitively, $\Omega X \simeq 0 \times_X 0$.
\end{definition}

\begin{definition}[Suspension Object]
For any object $X \in \C$, the suspension object $\Sigma X$ is the colimit of the diagram $0 \leftarrow X \to 0$. It fits into the following pushout square:
\[
\begin{tikzcd}
X \arrow[r] \arrow[d] \arrow[dr, phantom, "\ulcorner", very near end] & 0 \arrow[d] \\
0 \arrow[r] & \Sigma X
\end{tikzcd}
§\]
Intuitively, $\Sigma X \simeq 0 \amalg_X 0$.
\end{definition}
\begin{definition}[Stable $\infty$-Category]
An $\infty$-category $\C$ is called \textbf{stable} if it satisfies the following conditions:
\begin{enumerate}
    \item There exists a zero object $0 \in \C$ (i.e., $\C$ is pointed).
    \item Every morphism in $\C$ admits a kernel and a cokernel.
    \item A triangle in $\C$ is a pushout square if and only if it is a pullback square.
\end{enumerate}
\end{definition}
\begin{definition}[The Stabilization of an $\infty$-Category]
Let $\mathcal{C}$ be an $\infty$-category admitting finite limits. The process of constructing the stable $\infty$-category associated to $\mathcal{C}$ proceeds in two stages:

\begin{enumerate}
    \item \textbf{Pointed View (Formation of $\mathcal{C}_*$):} \\
    First, we construct the \textit{pointed} $\infty$-category $\mathcal{C}_*$. Assuming $\mathcal{C}$ has a terminal object $*$, $\mathcal{C}_*$ is defined as the under-category of the terminal object:
    \[
    \mathcal{C}_* := \mathcal{C}_{*/} \cong \text{Fun}(\Delta^1, \mathcal{C}) \times_{\text{Fun}(\{0\}, \mathcal{C})} \{*\}
    \]
    Objects in $\mathcal{C}_*$ are morphisms $* \to X$ in $\mathcal{C}$ (i.e., objects equipped with a base point). In $\mathcal{C}_*$, the object corresponding to the identity $* \to *$ serves as a \textit{zero object} (both initial and terminal). Consequently, the loop functor $\Omega: \mathcal{C}_* \to \mathcal{C}_*$ is well-defined by $\Omega X = * \times_X *$.

    \item \textbf{Stabilization (Formation of $\text{Sp}(\mathcal{C})$):} \\
    The \textit{stabilization} of $\mathcal{C}$, denoted as $\text{Sp}(\mathcal{C})$ (or $\text{Stab}(\mathcal{C})$), is defined as the $\infty$-category of spectrum objects in $\mathcal{C}_*$. It is constructed as the homotopy limit of the tower of loop functors:
    \[
    \text{Sp}(\mathcal{C}) := \varprojlim \left( \cdots \xrightarrow{\Omega} \mathcal{C}_* \xrightarrow{\Omega} \mathcal{C}_* \xrightarrow{\Omega} \mathcal{C}_* \right)
    \]
    Explicitly, an object $E \in \text{Sp}(\mathcal{C})$ consists of a sequence $\{E_n\}_{n \geq 0}$ of objects in $\mathcal{C}_*$ together with equivalences $E_n \xrightarrow{\sim} \Omega E_{n+1}$ for each $n$. This construction forces the suspension functor $\Sigma$ to be an equivalence, rendering $\text{Sp}(\mathcal{C})$ a stable $\infty$-category.
\end{enumerate}
\end{definition}

\begin{theorem}[Universal Property of Stabilization]
Let $\C$ be an $\infty$-category with finite limits and a terminal object $\ast$. Let $\C_* = \C_{\ast/}$ be its pointed version. The stabilization $\Sp(\C)$ is characterized by the following equivalent descriptions:

\begin{enumerate}
    \item \textbf{Internal Construction (Loop Towers):} 
    $\Sp(\C)$ is the homotopy limit of the sequence of loop functors:
    \[
    \Sp(\C) \simeq \varprojlim \left( \cdots \xrightarrow{\Omega} \C_* \xrightarrow{\Omega} \C_* \xrightarrow{\Omega} \C_* \right)
    \]
    An object in $\Sp(\C)$ is an $\Omega$-spectrum, i.e., a sequence $\{E_n\}_{n \geq 0}$ in $\C_*$ with equivalences $E_n \simeq \Omega E_{n+1}$.

    \item \textbf{External Construction (Excision):} 
    $\Sp(\C)$ is equivalent to the $\infty$-category of pointed excisive functors from the category of finite pointed spaces $\Sfin$ to $\C$:
    \[
    \Sp(\C) \simeq \Exc_*(\Sfin, \C)
    \]
    A functor $F: \Sfin \to \C$ belongs to $\Exc_*(\Sfin, \C)$ if $F(\ast) \simeq \ast$ and $F$ maps every pushout square in $\Sfin$ to a pullback square in $\C$.
\end{enumerate}

Furthermore, the stabilization $\Sp(\C)$ is the universal stable $\infty$-category under $\C$: for any stable $\infty$-category $\mathcal{D}$, the functor $\Sp(\C) \to \mathcal{D}$ induces an equivalence of $\infty$-categories $\text{Fun}^{\text{lex}}(\Sp(\C), \mathcal{D}) \simeq \text{Fun}^{\text{lex}}(\C, \mathcal{D})$, where $\text{Fun}^{\text{lex}}$ denotes the $\infty$-category of left exact functors.
\end{theorem}

\begin{remark}[Distinction between $\Stab(\C)$ and $\Sp$]
It is essential to distinguish between the abstract process of stabilization and the specific category of spectra:
\begin{enumerate}
    \item \textbf{The Category of Spectra ($\Sp$):} Historically and by convention, $\Sp$ refers specifically to the stabilization of the $\infty$-category of pointed spaces $\Sspace_{\ast}$. That is:
    \[ \Sp \simeq \Stab(\Sspace) \simeq \varprojlim (\dots \xrightarrow{\Omega} \Sspace_{\ast} \xrightarrow{\Omega} \Sspace_{\ast}) \]
    This category serves as the unit object in the $\infty$-category of stable $\infty$-categories and provides the ground for stable homotopy theory.

    \item \textbf{Stabilization of an Arbitrary $\infty$-Category ($\Stab(\C)$):} For any $\infty$-category $\C$ with finite limits, $\Stab(\C)$ is the stable $\infty$-category constructed as the limit of the tower of loop functors:
    \[ \Stab(\C) = \varprojlim (\dots \xrightarrow{\Omega} \C_{\ast} \xrightarrow{\Omega} \C_{\ast}) \]
    While $\Stab(\C)$ is always a stable $\infty$-category, it may not possess a symmetric monoidal structure (like the smash product) unless $\C$ itself is equipped with a compatible monoidal structure.

    \item \textbf{The Relation:} Any stable $\infty$-category $\mathcal{D}$ is naturally tensored over $\Sp$. In this sense, $\Sp$ plays a role analogous to the ring of integers $\mathbb{Z}$ in abelian groups: for any $D \in \mathcal{D}$ and $E \in \Sp$, there is a well-defined object $E \otimes D \in \mathcal{D}$.
\end{enumerate}
\end{remark}

\begin{definition}[Internal Mapping Spectrum in $\Stab(\C)$]
Let $\C$ be a \textbf{closed symmetric monoidal $\infty$-category} $(\C, \otimes, \mathbf{1})$ that admits finite limits. Assume further that the tensor product $\otimes$ is compatible with the stabilization (i.e., it preserves colimits in each variable). 

Let $\mathcal{D} = \Stab(\C)$ be the resulting stable $\infty$-category, equipped with the induced symmetric monoidal structure $\otimesD$. For any objects $X, Y \in \mathcal{D}$, the \textbf{internal mapping spectrum} is defined as the object $\uMap_{\mathcal{D}}(X, Y) \in \mathcal{D}$ that satisfies the following conditions:

\begin{enumerate}
    \item \textbf{Adjunction Property:} It is the right adjoint to the tensor product functor. For any object $Z \in \mathcal{D}$, there is a natural equivalence of mapping spaces:
    \[ \Map_{\mathcal{D}}(Z \otimesD X, Y) \simeq \Map_{\mathcal{D}}(Z, \uMap_{\mathcal{D}}(X, Y)) \]
    
    \item \textbf{Spectrum Level Structure:} In terms of the sequence of objects $\{E_n\} \in \C_{\ast}$ representing the spectrum, the $n$-th level of the internal mapping spectrum is given by:
    \[ \uMap_{\mathcal{D}}(X, Y)_n \simeq \uMap_{\C}(X, \Sigma^n Y) \]
    where $\uMap_{\C}$ denotes the internal Hom in the underlying category $\C$ (if it exists) or the corresponding enrichment.
\end{enumerate}
\end{definition}

\begin{definition}[Mapping Spectrum]
Let $X$ and $Y$ be spectra in $\Sp$. The \textbf{mapping spectrum} from $X$ to $Y$, denoted as $\uMap(X, Y) \in \Sp$, is the unique spectrum (up to equivalence) characterized by the following properties:

\begin{enumerate}
    \item \textbf{Adjunction (Internal Hom):}
    For any spectrum $Z$, there is a natural equivalence of mapping spaces:
    \[
    \Map_{\Sp}(Z \wedgeS X, Y) \simeq \Map_{\Sp}(Z, \uMap(X, Y))
    \]
    This identifies $\uMap(X, Y)$ as the right adjoint to the smash product functor $( - \wedgeS X )$.

    \item \textbf{Omega-Spectrum Structure:}
    The $n$-th space of the mapping spectrum is equivalent to the space of maps from $X$ to the $n$-th suspension of $Y$:
    \[
    \uMap(X, Y)_n \simeq \Map_{\Sp}(X, \Sigma^n Y)
    \]
    The structure maps $\Sigma \uMap(X, Y)_n \to \uMap(X, Y)_{n+1}$ are induced by the stability of $\Sp$.
\end{enumerate}
\end{definition}


\begin{definition}[Homotopy Groups in $\Stab(\C)$]
Let $\C$ be a closed symmetric monoidal $\infty$-category with finite limits, and let $\D = \Stab(\C)$ be its stabilization with unit object $\Unit$. 
\begin{enumerate}
    \item \textbf{Homotopy Groups of an Object:} For any object $E \in \D$ and $n \in \mathbb{Z}$, the $n$-th homotopy group of $E$ is defined as the abelian group of homotopy classes of maps from the $n$-shifted unit object:
    \[ \pi_n(E) := [\Sigma^n \Unit, E]_{\D}  \]
    
    \item \textbf{Homotopy Groups of the Mapping Spectrum:} Let $\uMap_{\D}(X, Y) \in \D$ be the internal mapping spectrum between $X, Y \in \D$. Its homotopy groups characterize the graded morphisms between the two objects:
    \[ \pi_n \uMap_{\D}(X, Y) \cong [X, \Sigma^n Y]_{\D} \cong [\Sigma^{-n} X, Y]_{\D} \]
    where $[-, -]_{\D}$ denotes the set of homotopy classes i.e. the $0$-th homotopy group of the kan complex $\Map_{\D}(-, -)$.
\end{enumerate}
\end{definition}

\begin{remark}
The distinction lies in the target category:
\begin{itemize}
    \item $\Map_{\D}(X, Y) \in \mathcal{S}$ is a \textbf{space} (Kan complex). It represents the mapping space in the $\infty$-categorical sense.
    \item $\uMap_{\D}(X, Y) \in \D$ is a \textbf{spectrum} (Internal Hom). It is an object of the stable category $\D$ that stabilizes the mapping space.
\end{itemize}
In terms of homotopy groups: $\pi_n \Map_{\D}(X, Y) \cong \pi_n \uMap_{\D}(X, Y)$ for $n \ge 0$, because every spectrum is $\Omega$-spectrum in $\Stab(\C)$.
\end{remark}

\begin{construction}[Stabilization of a Suspension Spectrum]
Let $X \in \Sspace_{\ast}$ be a pointed space (or generally an object in a pointed $\infty$-category $\mathcal{C}$ with finite colimits). The construction of its associated \textbf{suspension spectrum} $\SigInf X \in \Sp$ proceeds as follows:

\begin{enumerate}
    \item \textbf{The Prespectrum Construction:}
    First, we form a \textit{prespectrum} $P_X$ by iterating the suspension functor $\Sigma$ on $X$. This is given by the sequence of spaces:
    \[ (P_X)_n := \Sigma^n X, \quad \text{for } n \ge 0 \]
    together with the structural maps (identities):
    \[ \sigma_n: \Sigma((P_X)_n) = \Sigma(\Sigma^n X) \xrightarrow{id} \Sigma^{n+1} X = (P_X)_{n+1} \]

    \item \textbf{Spectrification (The L-functor):}
    Since $P_X$ is not necessarily an $\Omega$-spectrum (i.e., the adjoint maps $(P_X)_n \to \Omega (P_X)_{n+1}$ are not equivalences), we apply the \textit{spectrification functor} $L$ (or stabilization) to convert it into a true spectrum. The resulting object is the suspension spectrum:
    \[ \SigInf X := L(P_X) \]
    Conceptually, the $k$-th space of this stable object is the colimit:
    \[ (\SigInf X)_k \simeq \operatorname*{colim}_{m \to \infty} \Omega^m \Sigma^{m+k} X \]

    \item \textbf{Universal Property (Adjunction):}
    The construction defines the left adjoint functor $\SigInf$ in the stabilization adjunction:
    \[
    \begin{tikzcd}
        \Sspace_{\ast}\arrow[r, "\SigInf", bend left] & \Sp \arrow[l, "\OmeInf", bend left]
    \end{tikzcd}
    \]
    where for any spectrum $E$, the right adjoint is given by $\OmeInf E := E_0$ (the 0-th space of the $\Omega$-spectrum $E$).
\end{enumerate}
\end{construction}

\begin{remark}[Bousfield-Friedlander Structure and Stabilization]
    The Bousfield-Friedlander model structure $\mathcal{M}_{\mathrm{BF}}$ on the category of prespectra is the left Bousfield localization of the strict model structure $\mathcal{M}_{\mathrm{strict}}$. The three classes of morphisms in $\mathcal{M}_{\mathrm{BF}}$ are characterized as follows:
    \begin{itemize}
        \item \textbf{Cofibrations:} These are exactly the same as the strict cofibrations (levelwise inclusions that satisfy the appropriate cell complex conditions).
        \item \textbf{Weak Equivalences:} These are the \textit{stable weak equivalences}, i.e., maps $f: X \to Y$ that induce isomorphisms on stable homotopy groups $\pi_n^{S}(X) \cong \pi_n^{S}(Y)$ for all $n \in \mathbb{Z}$.
        \item \textbf{Fibrations:} These are the maps that satisfy the right lifting property with respect to acyclic cofibrations. Specifically, a map $p: E \to B$ is a BF-fibration if it is a levelwise fibration and the square 
        \[
        \begin{tikzcd}
            E_n \arrow[r] \arrow[d, "p_n"] & \Omega E_{n+1} \arrow[d, "\Omega p_{n+1}"] \\
            B_n \arrow[r] & \Omega B_{n+1}
        \end{tikzcd}
        \]
        is a homotopy pullback for all $n$.
    \end{itemize}
    
    The transition from $\mathcal{M}_{\mathrm{strict}}$ to $\mathcal{M}_{\mathrm{BF}}$ captures the essence of stabilization. Since the fibrant objects in this structure are exactly the $\Omega$-spectra, the \textbf{fibrant replacement} of a prespectrum $X$ in $\mathcal{M}_{\mathrm{BF}}$ is precisely its \textbf{stabilization} (spectrification). 
    
    If $R_{\mathrm{BF}}$ denotes the fibrant replacement functor, we have a natural stable equivalence $j: X \xrightarrow{\sim} R_{\mathrm{BF}}(X)$, where $R_{\mathrm{BF}}(X)$ is an $\Omega$-spectrum. In the stable homotopy category, this is equivalent to the classical stabilization $QX = \Omega^\infty \Sigma^\infty X$:
    \[
    \begin{tikzcd}
        X \arrow[r, "j", "\sim_{\text{stable}}"'] \arrow[d, equal] & R_{\mathrm{BF}}(X) \arrow[d, "\simeq"] \\
        X \arrow[r, "\text{Stabilization}"] & QX
    \end{tikzcd}
    \]
    Thus, the Bousfield-Friedlander model structure provides the formal homotopy-theoretic machinery where "becoming an $\Omega$-spectrum" is equivalent to "becoming fibrant."
\end{remark}

\begin{definition}
The \textbf{homotopy category functor} $ho$ is the change-of-base functor induced by the path-components functor $\pi_0: \mathbf{sSet} \to \mathbf{Set}$. For a category $\mathcal{C}$ enriched over simplicial sets, $ho(\mathcal{C})$ is the $\mathbf{Set}$-enriched category with the same objects as $\mathcal{C}$ and morphism sets defined by:
\[ \text{Hom}_{ho(\mathcal{C})}(X, Y) = \pi_0(\text{Map}_{\mathcal{C}}(X, Y)) \]
Composition in $ho(\mathcal{C})$ is inherited from the enriched composition in $\mathcal{C}$ via the product-preserving property of $\pi_0$.
\end{definition}

\begin{proposition}
Let $\mathcal{C}$ be a locally Kan simplicial category. Let $N_{\Delta}$ denote the homotopy coherent nerve and $N$ denote the classical nerve. There is a natural isomorphism of simplicial sets (or categories):
\[ ho(N_{\Delta}(\mathcal{C})) \cong N(ho(\mathcal{C})) \]
\end{proposition}

\begin{example}[The Stable Homotopy Category]
Let $\mathcal{S}p$ be the stable $\infty$-category of spectra. The classical \textbf{stable homotopy category} $\mathsf{SHC}$ is precisely its homotopy category:
\[ \mathsf{SHC} \cong ho(\mathcal{S}p) \]
Under the $ho$ functor, the enriched mapping spaces $\text{Map}_{\mathcal{S}p}(X, Y)$ are replaced by their sets of path-components $\pi_0$. The stable property of $\mathcal{S}p$ (the equivalence of fiber and cofiber sequences) ensures that $ho(\mathcal{S}p)$ inherits the structure of a \textbf{triangulated category}.
\end{example}


\endgroup}

\begingroup 

\subsection{$\infty$-Operad }

\begin{definition}[$\infty$-Operator Category]
An \textbf{$\infty$-operator category} is an $\infty$-category $\mathcal{B}$ equipped with a specified subcategory of \textit{inert morphisms} $\mathcal{B}^{\mathrm{inert}}$, satisfying the following two structural axioms illustrated by commutative diagrams:

\begin{enumerate}
    \item \textbf{Active-Inert Factorization:} 
    There exists a class of \textit{active morphisms} such that every morphism $f: X \to Z$ in $\mathcal{B}$ factors essentially uniquely as an active morphism followed by an inert morphism.
    \[
    \begin{tikzcd}
        X \arrow[rr, "f"] \arrow[dr, "f^{\mathrm{act}} \text{ (Mix)}"'] & & Z \\
         & Y \arrow[ur, "f^{\mathrm{inert}} \text{ (Select)}"'] &
    \end{tikzcd}
    \]
    Here, $f^{\mathrm{act}}$ performs the operations (combining inputs), and $f^{\mathrm{inert}}$ performs the structural projection.

    \item \textbf{Elementary Decomposition:} 
    There exists a set of \textit{elementary objects} (or colors) $\{U_\alpha\}$. Every object $X \in \mathcal{B}$ is determined by its inert projections to these elementary objects. Specifically, the collection of inert morphisms $\{\rho^i\}$ exhibits $X$ as a product:
    \[
    \begin{tikzcd}[column sep=small]
        & X \arrow[dl, "\rho^1"'] \arrow[d, "\rho^2"] \arrow[dr, "\rho^n"] & \\
        U_{i_1} & U_{i_2} & U_{i_n}
    \end{tikzcd}
    \]
    This implies an equivalence $X \simeq U_{i_1} \times U_{i_2} \times \dots \times U_{i_n}$, ensuring that complex objects are merely aggregates of elementary slots.
\end{enumerate}
\end{definition}

\begin{definition}[$\infty$-Operator Category]
An \textbf{$\infty$-operator category} is an $\infty$-category $\mathcal{B}$ equipped with a specified factorization structure, consisting of two subcategories: \textit{active morphisms} ($\mathcal{B}^{\mathrm{act}}$) and \textit{inert morphisms} ($\mathcal{B}^{\mathrm{inert}}$). These satisfy the following two axioms:

\begin{enumerate}
    \item \textbf{Active-Inert Factorization System:} 
    The pair $(\mathcal{B}^{\mathrm{act}}, \mathcal{B}^{\mathrm{inert}})$ forms an \textit{orthogonal factorization system} on $\mathcal{B}$. 
    
    This means that every morphism $f: X \to Z$ in $\mathcal{B}$ factors essentially uniquely as an active morphism followed by an inert morphism:
    \[
    \begin{tikzcd}[column sep=large, row sep=large]
        X \arrow[rr, "f"] \arrow[dr, "f^{\mathrm{act}} "'] & & Z \\
         & Y \arrow[ur, "f^{\mathrm{inert}}"'] &
    \end{tikzcd}
    \]
    Here, the intermediate object $Y$ and the active map $f^{\mathrm{act}}$ are not arbitrary; they are \textbf{determined} by the image of $f$ under this factorization system. 

    \item \textbf{Elementary Decomposition (Segal Core):} 
    There exists a set of elementary objects $\mathcal{E} \subset \mathrm{Ob}(\mathcal{B})$. For any object $X \in \mathcal{B}$, let $\Lambda_X$ be the set of all inert morphisms targeting $\mathcal{E}$:
    \[
        \Lambda_X := \{ \rho: X \to U \mid \rho \in \mathcal{B}^{\mathrm{inert}}, U \in \mathcal{E} \}
    \]
    We require that the canonical map induced by these morphisms is an equivalence:
    \[
        X \xrightarrow{\quad \simeq \quad} \prod_{\rho \in \Lambda_X} \mathrm{codom}(\rho)
    \]
\end{enumerate}

\end{definition}


\begin{definition}[$\mathcal{B}$-Operad]
Let $\mathcal{B}$ be an $\infty$-operator category. 
A \textbf{$\mathcal{B}$-operad} is a map of simplicial sets $p: \mathcal{C}^\otimes \to \mathcal{B}$ satisfying the following three conditions:

\begin{enumerate}
    \item \textbf{Inner Fibration:} The map $p$ is an inner fibration of simplicial sets. That is, for every $0 < k < n$, $p$ has the right lifting property with respect to the inner horn inclusion $\Lambda^n_k \hookrightarrow \Delta^n$.
    
    \item \textbf{Inert Lifting Property:} For every inert morphism $f: X \to Y$ in $\mathcal{B}$ and every object $C \in \mathcal{C}^\otimes$ such that $p(C) = X$, there exists a $p$-coCartesian edge $\bar{f}: C \to C'$ in $\mathcal{C}^\otimes$ such that $p(\bar{f}) = f$.
    
    \item \textbf{Segal Condition:} For every object $X \in \mathcal{B}$, let $\{X \xrightarrow{f_i} U_i\}_{i \in I}$ be the collection of inert morphisms decomposing $X$ into elementary objects (as dictated by the structure of $\mathcal{B}$). The functor induced by the $p$-coCartesian lifts of these inert morphisms,
    \[
        \mathcal{C}^\otimes_X \xrightarrow{\quad \simeq \quad} \prod_{i \in I} \mathcal{C}^\otimes_{U_i},
    \]
    is an equivalence of $\infty$-categories.
\end{enumerate}
\end{definition}

\begin{definition}[$\mathcal{O}$-Algebra Object]
Let $\mathcal{B}$ be an $\infty$-operator category. Let $p: \mathcal{O}^\otimes \to \mathcal{B}$ and $q: \mathcal{C}^\otimes \to \mathcal{B}$ be \textbf{$\mathcal{B}$-operad}.
An \textbf{$\mathcal{O}$-algebra object in $\mathcal{C}$} is a map of $\infty$-operads over $\mathcal{B}$. Explicitly, it is a functor
\[
A: \mathcal{O}^\otimes \longrightarrow \mathcal{C}^\otimes
\]
satisfying two conditions:

\begin{enumerate}
    \item \textbf{Commutativity over Base (Compatibility):} 
    The functor $A$ respects the projection to the base category $\mathcal{B}$. The following diagram commutes:
    \[
    \begin{tikzcd}[row sep=large, column sep=small]
        \mathcal{O}^\otimes \arrow[rr, "A"] \arrow[dr, "p"'] & & \mathcal{C}^\otimes \arrow[dl, "q"] \\
         & \mathcal{B} &
    \end{tikzcd}
    \]
    (i.e., $q \circ A = p$).

    \item \textbf{Inert Preservation (The Operad Map Condition):} 
    The functor $A$ carries inert morphisms to inert morphisms.
    
    Specifically, if $f$ is an \textit{inert morphism} in $\mathcal{O}^\otimes$ (meaning $f$ is a $p$-coCartesian lift of an inert map in $\mathcal{B}$), then its image $A(f)$ must be an \textit{inert morphism} in $\mathcal{C}^\otimes$ (meaning $A(f)$ is a $q$-coCartesian lift of that same map in $\mathcal{B}$).
\end{enumerate}

The $\infty$-category of all such algebras is denoted by $\mathrm{Alg}_{\mathcal{O}}(\mathcal{C})$.

\end{definition}

\begin{example}[\textbf{The Zoo of Operads and Their Bases}]
We classify common algebraic structures by specifying the underlying \textbf{Base Category} $\mathcal{B}$ (which dictates the geometry of inputs) and the \textbf{Operad} $\mathcal{O}^\otimes$ (which dictates the operations) as a fibration $p: \mathcal{O}^\otimes \to \mathcal{B}$.

\begin{enumerate}
    \item \textbf{The Commutative Case ($E_\infty$)}
    \begin{itemize}
        \item \textbf{Base:} $\mathcal{B} = N(\mathrm{Fin}_*)$ (Symmetric/Unordered inputs).
        \item \textbf{Operad:} $\mathcal{O}^\otimes = \mathrm{Comm}^\otimes := N(\mathrm{Fin}_*)$.
        \item \textbf{Structure Map:} The identity map $\mathrm{id}: N(\mathrm{Fin}_*) \to N(\mathrm{Fin}_*)$.
        \item \textbf{Resulting Algebra:} \textbf{Commutative $\infty$-Algebra}.
        \item \textbf{Note:} Since the map is the identity, the fiber over any operation is a point. There is essentially only one way to combine inputs (order doesn't matter).
    \end{itemize}

    \item \textbf{The Associative Case ($A_\infty$)}
    \begin{itemize}
        \item \textbf{Base:} $\mathcal{B} = N(\mathrm{Fin}_*)$.
        \item \textbf{Operad:} $\mathcal{O}^\otimes = \mathrm{Ass}^\otimes$ (The Associative Operad).
        \item \textbf{Structure Map:} The "forgetful" functor that forgets the linear ordering of the fibers.
        \item \textbf{Resulting Algebra:} \textbf{Associative $\infty$-Algebra}.
        \item \textbf{Note:} The fiber over $\langle n \rangle \to \langle 1 \rangle$ is equivalent to the symmetric group $\Sigma_n$. This allows inputs to be permuted (by the base), but the operation distinguishes the order of multiplication ($x_1 x_2 \neq x_2 x_1$).
    \end{itemize}

    \item \textbf{The Little $k$-Disks Case ($E_k$)}
    \begin{itemize}
        \item \textbf{Base:} $\mathcal{B} = N(\mathrm{Fin}_*)$.
        \item \textbf{Operad:} $\mathcal{O}^\otimes = \mathbb{E}_k^\otimes$.
        \item \textbf{Structure Map:} The projection from the space of disk embeddings.
        \item \textbf{Resulting Algebra:} \textbf{$E_k$-Algebra}.
        \item \textbf{Note:} Interpolates between Associative ($k=1$) and Commutative ($k=\infty$).
    \end{itemize}

    \item \textbf{The Lie Case ($L_\infty$)}
    \begin{itemize}
        \item \textbf{Base:} $\mathcal{B} = N(\mathrm{Fin}_*)$.
        \item \textbf{Operad:} $\mathcal{O}^\otimes = \mathrm{Lie}^\otimes$.
        \item \textbf{Resulting Algebra:} \textbf{$L_\infty$-Algebra} (Homotopy Lie Algebra).
        \item \textbf{Note:} Typically considered over a stable target category (like chain complexes).
    \end{itemize}

    \item \textbf{The Non-Symmetric / Planar Case}
    \begin{itemize}
        \item \textbf{Base:} $\mathcal{B} = N(\Delta)^{op}$ (The Simplex Category; Linear/Ordered inputs).
        \item \textbf{Operad:} $\mathcal{O}^\otimes = N(\Delta)^{op}$.
        \item \textbf{Structure Map:} The identity map.
        \item \textbf{Resulting Algebra:} \textbf{Associative Monoid} (in the strict sense).
        \item \textbf{Note:} Here, the base category itself forbids permutation. There is no symmetric group action to even consider.
    \end{itemize}
\end{enumerate}


\end{example}

\begin{definition}[Endomorphism $\infty$-Category]
Let $\mathcal{C}$ be an $\infty$-category. The \textbf{Endomorphism $\infty$-Category}, denoted by $\text{End}(\mathcal{C})$, is defined as the functor $\infty$-category $\text{Fun}(\mathcal{C}, \mathcal{C})$. 

It forms a \textbf{monoidal $\infty$-category} where the monoidal structure is determined by the composition of endofunctors:
\begin{enumerate}
    \item The tensor product is given by composition: $F \otimes G := F \circ G$.
    \item The unit object is given by the identity functor: $\mathbb{I} := \text{Id}_{\mathcal{C}}$.
\end{enumerate}
\end{definition}



\begin{definition}[Spaces of Monads and Comonads]
Let $\mathcal{C}$ be an $\infty$-category, and let $\text{End}(\mathcal{C})$ denote the monoidal $\infty$-category of endofunctors on $\mathcal{C}$ (equipped with the composition product).
The $\infty$-categories (or spaces) of Monads and Comonads as the categories of associative algebra objects in $\text{End}(\mathcal{C})$ and its opposite, respectively:

\begin{enumerate}
    \item The \textbf{$\infty$-category of Monads} is defined as:
    \[
    \text{Mnd}(\mathcal{C}) := \text{Alg}_{\mathcal{A}ss}(\text{End}(\mathcal{C}))
    \]
    
    \item The \textbf{$\infty$-category of Comonads} is defined as:
    \[
    \text{CoMnd}(\mathcal{C}) := \text{Alg}_{\mathcal{A}ss}(\text{End}(\mathcal{C})^{op})
    \]
\end{enumerate}
The objects of $\text{Mnd}(\mathcal{C})$ are referred to as \textbf{Monads} on $\mathcal{C}$, and the objects of $\text{CoMnd}(\mathcal{C})$ are referred to as \textbf{Comonads} on $\mathcal{C}$.
\end{definition}

\begin{definition}[Reedy Category]
A small category $\mathcal{R}$ is a \textbf{Reedy category} if it is equipped with a degree function $d: \text{Ob}(\mathcal{R}) \to \lambda$ (where $\lambda$ is an ordinal) and two subcategories $\overrightarrow{\mathcal{R}}$ (the direct category) and $\overleftarrow{\mathcal{R}}$ (the inverse category), such that:
\begin{enumerate}
    \item Every non-identity morphism in $\overrightarrow{\mathcal{R}}$ raises the degree.
    \item Every non-identity morphism in $\overleftarrow{\mathcal{R}}$ lowers the degree.
    \item Every morphism $f$ in $\mathcal{R}$ factors uniquely as $f = g \circ h$, where $h \in \overleftarrow{\mathcal{R}}$ and $g \in \overrightarrow{\mathcal{R}}$.
\end{enumerate}
\end{definition}

\begin{definition}[The Reedy Model Structure]
Let $\mathcal{M}$ be a model category and $\mathcal{R}$ be a Reedy category. The category of diagrams $\text{Fun}(\mathcal{R}, \mathcal{M})$ is equipped with the \textbf{Reedy model structure}, where a morphism $f: X \to Y$ is defined to be:

\begin{enumerate}
    \item A \textbf{Weak Equivalence} if it is a levelwise weak equivalence. That is, for every object $\alpha \in \mathcal{R}$, the map $f_\alpha: X_\alpha \to Y_\alpha$ is a weak equivalence in $\mathcal{M}$.

    \item A \textbf{Cofibration} if for every $\alpha \in \mathcal{R}$, the \textbf{relative latching map} $\lambda_\alpha(f)$ is a cofibration in $\mathcal{M}$.
    Here, $\lambda_\alpha(f)$ is the map induced by the pushout of the latching objects:
    \[
    \lambda_\alpha(f): X_\alpha \amalg_{L_\alpha X} L_\alpha Y \longrightarrow Y_\alpha
    \]
    where the \textit{latching object} is defined as $L_\alpha X = \text{colim}_{\partial \vec{\mathcal{R}}/\alpha} X$.

    \item A \textbf{Fibration} if for every $\alpha \in \mathcal{R}$, the \textbf{relative matching map} $\mu_\alpha(f)$ is a fibration in $\mathcal{M}$.
    Here, $\mu_\alpha(f)$ is the map induced into the pullback of the matching objects:
    \[
    \mu_\alpha(f): X_\alpha \longrightarrow M_\alpha X \times_{M_\alpha Y} Y_\alpha
    \]
    where the \textit{matching object} is defined as $M_\alpha X = \lim_{\alpha / \partial \overleftarrow{\mathcal{R}}} X$.
\end{enumerate}
\end{definition}

\begin{proposition}[Latching and Matching as Monadic Structures]
Let $\mathcal{R}$ be a Reedy category. For any degree $\alpha$, consider the truncation inclusion of the category of degrees strictly lower than $\alpha$:
\[ u: \mathcal{R}_{< \alpha} \hookrightarrow \mathcal{R}_{\le \alpha} \]
Let $u^*$ be the restriction functor. We identify the Latching and Matching objects via the adjunctions defining the skeleton and coskeleton:

\begin{enumerate}
    \item \textbf{Latching as a Monad (Skeleton):}
    The pair $u_! \dashv u^*$ generates a \textbf{Monad} $T = u_! \circ u^*$ (Left Kan Extension followed by restriction). The Latching object is the value of this monad:
    \[
    L_\alpha X \cong (T(X))_\alpha = (\text{Lan}_u (u^* X))_\alpha
    \]
    The canonical map $L_\alpha X \to X_\alpha$ corresponds to the monad algebra structure map (or the counit of the adjunction).

    \item \textbf{Matching as a Comonad (Coskeleton):}
    The pair $u^* \dashv u_*$ generates a \textbf{Comonad} $G = u_* \circ u^*$ (Right Kan Extension followed by restriction). The Matching object is the value of this comonad:
    \[
    M_\alpha X \cong (G(X))_\alpha = (\text{Ran}_u (u^* X))_\alpha
    \]
    The canonical map $X_\alpha \to M_\alpha X$ corresponds to the comonad coalgebra structure map (or the unit of the adjunction).
\end{enumerate}

\end{proposition}

\begin{definition}[Monadic and Comonadic Resolution]
Let $\mathcal{C}$ be a category and $X$ an object in $\mathcal{C}$. We define the canonical resolutions generated by monads and comonads as follows:

\begin{enumerate}
    \item \textbf{Monadic Resolution:}
    Let $(T, \mu, \eta)$ be a \textbf{Monad} on $\mathcal{C}$. The \textbf{Bar Construction} provides an augmented simplicial object $B_\bullet(X)$ resolving $X$:
    \[
    \begin{tikzcd}
    \dots \arrow[r, shift left] \arrow[r, shift right] 
    & T^3 X \arrow[r, shift left] \arrow[r] \arrow[r, shift right] 
    & T^2 X \arrow[r, shift left] \arrow[r, shift right] 
    & TX \arrow[r, "\epsilon"] 
    & X
    \end{tikzcd}
    \]
    The face maps $d_i$ are given by the multiplication $\mu$, and degeneracy maps $s_i$ by the unit $\eta$. This construction typically serves as a \textbf{cofibrant replacement} of $X$.

    \vspace{1em}

    \item \textbf{Comonadic Resolution:}
    Let $(G, \delta, \epsilon)$ be a \textbf{Comonad} on $\mathcal{C}$. The \textbf{Cobar Construction} provides an augmented cosimplicial object $C^\bullet(X)$ resolving $X$:
    \[
    \begin{tikzcd}
    X \arrow[r, "\eta"] 
    & GX \arrow[r, shift left] \arrow[r, shift right] 
    & G^2 X \arrow[r, shift left] \arrow[r] \arrow[r, shift right] 
    & G^3 X \arrow[r, shift left] \arrow[r, shift right] 
    & \dots
    \end{tikzcd}
    \]
    The coface maps $d^i$ are given by the comultiplication $\delta$, and codegeneracy maps $s^i$ by the counit $\epsilon$. This construction typically serves as a \textbf{fibrant replacement} of $X$.
\end{enumerate}

\end{definition}


\begin{example}[\textbf{Reedy Latching and Matching}]
In a Reedy category $\mathcal{R}$, resolutions arise from Kan extensions along the filtration $u: \mathcal{R}_{< \alpha} \hookrightarrow \mathcal{R}_{\le \alpha}$.
\begin{itemize}
    \item \textbf{Latching (Monad):} The Latching object $L_\alpha X$ is generated by the \textbf{Skeleton Monad} $T = u_! u^*$ (Left Kan extension followed by restriction).
    \[ L_\alpha X \cong (T X)_\alpha \]
    \item \textbf{Matching (Comonad):} The Matching object $M_\alpha X$ is generated by the \textbf{Coskeleton Comonad} $G = u_* u^*$ (Right Kan extension followed by restriction).
    \[ M_\alpha X \cong (G X)_\alpha \]
\end{itemize}
\end{example}

\begin{example}[\textbf{The Cotangent Complex (André-Quillen)}]
Used to define the derived cotangent complex $\mathbb{L}_{A/k}$.
\begin{itemize}
    \item \textbf{Monad:} The \textbf{Free Algebra Monad} $T$ on the category of $k$-modules (or sets).
    \[ T(V) = \text{Sym}_k(V) \quad (\text{Polynomial Algebra}) \]
    \item \textbf{Resolution:} The simplicial resolution $P_\bullet \to A$ is the Bar construction $B_\bullet(T, T, A)$. The cotangent complex is derived from applying differentials $\Omega^1_{P_\bullet/k} \otimes_{P_\bullet} A$.
\end{itemize}
\end{example}

\begin{example}[\textbf{The Postnikov Tower}]
Decomposing a space $X$ into its homotopy types.
\begin{itemize}
    \item \textbf{Monad:} The \textbf{$n$-Truncation Monad} $\tau_{\le n}$ (or $P_n$).
    \[ T_n(X) = \tau_{\le n}(X) \]
    This is an \textit{idempotent} monad (localization). The tower is the limit sequence $\dots \to T_n X \to T_{n-1} X$.
    \item \textit{Note:} Dually, the \textbf{Whitehead Tower} uses the $n$-connected cover Comonad $\tau_{>n}$.
\end{itemize}
\end{example}

\begin{example}[\textbf{Projective and Injective Resolution}]
Let $R$ be a ring and $M$ an $R$-module.

\begin{enumerate}
    \item \textbf{Projective Resolution (The Bar Construction):}
    Using the free-forgetful adjunction $F \dashv U$, we define the \textbf{Free Monad} $T = F \circ U$. Since $T(M)$ is a free module, the associated Bar construction yields a canonical projective resolution:
    \[
    \dots \longrightarrow T^3 M \longrightarrow T^2 M \longrightarrow T M \xrightarrow{\epsilon} M \longrightarrow 0
    \]
    The boundary maps are alternating sums of the monad multiplication $\mu: T^2 \to T$.

    \item \textbf{Injective Resolution (The Cobar Construction):}
    Using the forgetful-cofree adjunction $U \dashv C$ (where $C(A) = \text{Hom}_{\mathbb{Z}}(R, A)$), we define the \textbf{Cofree Comonad} $G = C \circ U$. Since $G(M)$ is an injective module, the associated Cobar construction yields a canonical injective resolution:
    \[
    0 \longrightarrow M \xrightarrow{\eta} G M \longrightarrow G^2 M \longrightarrow G^3 M \longrightarrow \dots
    \]
    The boundary maps are alternating sums of the comonad comultiplication $\delta: G \to G^2$.
\end{enumerate}

\end{example}

\begin{example}[\textbf{The Spectrification Monad on Prespectra}]
Let $\mathcal{P}$ be the category of Prespectra (sequences of spaces with maps $\Sigma E_n \to E_{n+1}$). The process of converting a naive suspension spectrum into a genuine $\Omega$-spectrum is governed by the \textbf{Spectrification Monad} $\mathbb{L}$.

\begin{enumerate}
    \item \textbf{The Level-wise Monad:}
    We define a Monad $\mathbb{L}: \mathcal{P} \to \mathcal{P}$ by applying the spatial stabilization monad $Q = \Omega^\infty \Sigma^\infty$ to \textit{each level} of the spectrum independently:
    \[ (\mathbb{L} E)_n := Q(E_n) = \mathop{\text{colim}}_{k} \Omega^k \Sigma^k E_n \]
    
    \item \textbf{Application to Suspension Spectra:}
    If $E = \Sigma^\infty X$ is the suspension spectrum of $X$ (where $E_n = \Sigma^n X$), applying this monad yields:
    \[ (\mathbb{L}(\Sigma^\infty X))_n = Q(\Sigma^n X) \]
    The result $\mathbb{L}(\Sigma^\infty X)$ is an \textbf{$\Omega$-spectrum}. This is the \textit{fibrant replacement} of $\Sigma^\infty X$ in the stable model structure.

    \item \textbf{Distinction from Adams Resolution:}
    \begin{itemize}
        \item The \textbf{Adams/Bousfield-Kan resolution} builds a tower $X \to QX \to Q^2 X \dots$ to resolve the \textit{space} $X$.
        \item The \textbf{Spectrification Monad} $\mathbb{L}$ acts once (essentially as a completion) to fix the \textit{structure} of the spectrum, ensuring the adjoint structure maps $E_n \to \Omega E_{n+1}$ become weak equivalences.
    \end{itemize}
\end{enumerate}

\end{example}

\begin{example}[\textbf{The $R$-Completion of a Space}]
Let $R$ be a commutative ring (typically $\mathbb{Z}_p$ or $\mathbb{Q}$). The Bousfield-Kan resolution constructs the "$R$-completion" of a space $X$, effectively translating algebraic information (homology with coefficients in $R$) into homotopy information.

\begin{enumerate}
    \item \textbf{The Monad ($R$-Linearization):}
    Let $R: \mathcal{S} \to \mathcal{S}$ be the monad that assigns to a simplicial set $K$ the free simplicial $R$-module generated by $K$ (forgetting the module structure back to a simplicial set).
    \[ X \xrightarrow{\eta} R(X) \]
    Intuitively, this replaces every simplex of $X$ with the free $R$-module generated by its vertices.

    \item \textbf{The Cosimplicial Space (Bar Construction):}
    Applying the Monad iteratively generates a \textbf{cosimplicial space} $R^\bullet X$:
    \[
    \begin{tikzcd}
    X \arrow[r, "\eta"] 
    & R(X) \arrow[r, shift left] \arrow[r, shift right] 
    & R(R(X)) \arrow[r, shift left] \arrow[r] \arrow[r, shift right] 
    & R^3(X) \cdots
    \end{tikzcd}
    \]
    This tower resolves $X$ by spaces that are algebraically simple (generalized Eilenberg-MacLane spaces).

    \item \textbf{Totalization ($R$-Completion):}
    The \textbf{Totalization} (Homotopy Limit) of this cosimplicial space defines the $R$-completion of $X$:
    \[ X^\wedge_R := \text{Tot}(R^\bullet X) \simeq \mathop{\text{holim}}_{\Delta} R^\bullet X \]
    
    \item \textbf{The Spectral Sequence:}
    This resolution yields the \textbf{Bousfield-Kan Spectral Sequence}, which computes the homotopy groups of the completion from the cohomology of $X$:
    \[ E_2^{s,t} \cong \text{Ext}^s_{\text{Comod}}(R, H_*(X; R))_t \implies \pi_{t-s}(X^\wedge_R) \]
    For $R=\mathbb{Z}_p$, this computes the homotopy groups of the $p$-adic completion of $X$.
\end{enumerate}

\end{example}
\endgroup