%----------------------------------------------------------------------------------------
%	SECTION 2: Preliminaries
%----------------------------------------------------------------------------------------
{\begingroup


% Categories & Functors
\newcommand{\QCoh}{\operatorname{QCoh}}
\newcommand{\Coh}{\operatorname{Coh}}
\newcommand{\Ind}{\operatorname{Ind}}
\newcommand{\Fun}{\operatorname{Fun}}
\newcommand{\Map}{\operatorname{Map}} % Mapping space
\newcommand{\End}{\operatorname{End}}
\newcommand{\Mod}{\operatorname{Mod}}
\newcommand{\Alg}{\operatorname{Alg}}
\newcommand{\CAlg}{\operatorname{CAlg}}
\newcommand{\Sp}{\operatorname{Sp}} % Spectra
\newcommand{\PSh}{\mathcal{P}}      % Presheaves

% Geometry
\newcommand{\Spec}{\operatorname{Spec}}
\newcommand{\Sch}{\operatorname{Sch}}
\newcommand{\St}{\operatorname{St}} 
\newcommand{\co}{\mathcal{O}} % Structure sheaf

% Limits and Colimits
\newcommand{\limit}{\varprojlim}
\newcommand{\colimit}{\varinjlim}
\newcommand{\tensor}{\otimes}

% Logic & Sets
\newcommand{\infcat}{$\infty$-category}
\newcommand{\infcats}{$\infty$-categories}
\newcommand{\sSet}{\operatorname{Set}_{\Delta}} % Simplicial Sets

\newcommand{\Cat}{\operatorname{Cat}}            % 经典 1-范畴构成的范畴
\newcommand{\CatI}{\mathcal{C}\mathrm{at}_{\infty}} % 无穷范畴构成的无穷范畴
\newcommand{\Kan}{\mathbf{Kan}}                  % Kan 复形全子范畴
\newcommand{\Ob}{\mathrm{Ob}}  

\newcommand{\C}{\mathcal{C}}
\newcommand{\Sfin}{\mathcal{S}}
\newcommand{\Nerve}{N}
\newcommand{\Rig}{\mathfrak{C}}
\newcommand{\Pinf}{\mathcal{P}}
\newcommand{\Psimp}{\mathcal{P}_{\Delta}}

\section{Preliminaries}
\subsection{$\infty$-categories}


\begin{definition}[$\infty$-Category]
A simplicial set $K$ is an \textbf{$\infty$-category} if for every $n > 1$ and every \textbf{inner} index $0 < i < n$, every map of simplicial sets $f_0: \Lambda^n_i \to K$ admits an extension to an $n$-simplex $f: \Delta^n \to K$.
\[
\begin{tikzcd}
\Lambda^n_i \arrow[d, hook] \arrow[r, "f_0"] & K \\
\Delta^n \arrow[ru, dashed, "f"'] & 
\end{tikzcd}
\]
\end{definition}

\begin{definition}[Simplicial Category]
A \textbf{simplicial category} (or $\sSet$-enriched category) $\mathcal{C}$ is a category where:
\begin{enumerate}
    \item For any two objects $X, Y \in \mathcal{C}$, the collection of morphisms between them is not a set, but a \textbf{simplicial set} $\Map_{\mathcal{C}}(X, Y)$.
    \item For any three objects $X, Y, Z \in \mathcal{C}$, the composition map 
    \[ \Map_{\mathcal{C}}(Y, Z) \times \Map_{\mathcal{C}}(X, Y) \to \Map_{\mathcal{C}}(X, Z) \]
    is a morphism of simplicial sets and satisfies the usual associativity and identity axioms.
\end{enumerate}
A simplicial category $\mathcal{C}$ is \textbf{locally Kan} if for every pair of objects $X, Y \in \Ob(\mathcal{C})$, the mapping simplicial set $\Map_{\mathcal{C}}(X, Y)$ is a Kan complex.
\end{definition}




\begin{definition}[Simplicial Nerve $N_{\Delta}$]
The \textbf{simplicial nerve} $N_{\Delta}(\mathcal{C})$ is the simplicial set defined by the assignment:
\[ 
    N_{\Delta}(\mathcal{C})_n = \mathrm{Hom}_{\Cat_{\Delta}}(\mathfrak{C}[\Delta^n], \mathcal{C}) 
\]
where $\mathfrak{C}[\Delta^n]$ is the \textbf{rigidification} of the $n$-simplex $\Delta^n$ into a simplicial category.
\end{definition}

\begin{definition}[$\infty$-category via $N_\Delta$]
An \textbf{$\infty$-category} (or quasicategory) is a simplicial set $K$ that is equivalent to the simplicial nerve of some locally Kan simplicial category $\mathcal{C}$.
\[ K \simeq N_{\Delta}(\mathcal{C}) \]
\end{definition}





\begin{theorem}[Joyal-Lurie]
There exists a Quillen equivalence between the Joyal model structure on $\sSet$ (modeling quasicategories) and the Bergner model structure on $\Cat_{\Delta}$ (modeling simplicial categories):
\[
\mathfrak{C}[\cdot] : \sSet \rightleftarrows \Cat_{\Delta} : N_{\Delta}.
\]
Specifically, for any simplicial category $\mathcal{C}$ where mapping spaces are Kan complexes, its simplicial nerve $N_{\Delta}(\mathcal{C})$ is a quasicategory.
\end{theorem}




\begin{definition}[Free Cocompletion]
Let $\mathcal{C}$ be a small \infcat{}. An \infcat{} $\PSh(\mathcal{C})$ is called the \textbf{free cocompletion} of $\mathcal{C}$ if it satisfies the following universal property:

\begin{enumerate}
    \item $\PSh(\mathcal{C})$ admits all small colimits.
    \item There exists a functor $j: \mathcal{C} \to \PSh(\mathcal{C})$ (called the Yoneda embedding) such that for any \infcat{} $\mathcal{D}$ which admits small colimits, composition with $j$ induces an equivalence of \infcats{}:
    \[
        \Fun^L(\PSh(\mathcal{C}), \mathcal{D}) \xrightarrow{\sim} \Fun(\mathcal{C}, \mathcal{D}).
    \]
\end{enumerate}
Here, $\Fun^L$ denotes the full subcategory of functors that preserve small colimits (left adjoints).
\end{definition}


\begin{definition}[The \infcat{} of Spaces]
Let $\mathcal{S}$ denote the \infcat{} of spaces. It is defined in two equivalent ways:

\begin{enumerate}
    \item \textbf{Via Dwyer-Kan Localization:} \\
    Let $W$ be the class of weak homotopy equivalences in $\sSet$. We define $\mathcal{S}$ as the homotopy coherent nerve of the simplicial localization:
    \[
        \mathcal{S} \coloneqq N\bigl(\sSet[W^{-1}]\bigr).
    \]
    Equivalently, via Kan complexes: $\mathcal{S} \simeq N(\Kan)$.

    \item \textbf{Via Free Cocompletion:} \\
    The \infcat{} $\mathcal{S}$ is the free cocompletion of the point $*$. That is, it is the category of presheaves:
    \[
        \mathcal{S} \simeq \PSh(*).
    \]
    Universal Property: For any cocomplete \infcat{} $\mathcal{C}$, there is an equivalence $\Fun^L(\mathcal{S}, \mathcal{C}) \simeq \mathcal{C}$.
\end{enumerate}
\end{definition}

\begin{theorem}
Let $K$ be an $\infty$-category (quasi-category). Let $\C = \Rig[K]$ be its associated simplicial category (via rigidification). The construction of the presheaf $\infty$-category commutes with the nerve construction in the following sense:

\begin{enumerate}
    \item \textbf{Simplicial Side:} Consider the category of simplicial presheaves $\Psimp(\C) := \mathrm{Fun}_{\Delta}(\C^{op}, \Sfin_{\text{Kan}})$. This category admits a simplicial model structure (projective structure).
    
    \item \textbf{Infinity Side:} Consider the $\infty$-category of presheaves $\Pinf(K) := \mathrm{Fun}(K^{op}, \Sfin)$.
    
    \item \textbf{Equivalence:} There is an equivalence of $\infty$-categories:
    \[ \Pinf(K) \simeq \Nerve \left( \Psimp(\C)^{\mathrm{cf}} \right) \]
    where $\Psimp(\C)^{\mathrm{cf}}$ denotes the full simplicial subcategory of fibrant-cofibrant objects in the model category of simplicial presheaves.
\end{enumerate}
In summary, the presheaf of an $\infty$-category is modeled by the nerve of the strictly cocomplete simplicial category of enriched presheaves.
\end{theorem}

\begin{remark}[Homotopy Category via Fibrant-Cofibrant Objects]
To correctly construct the homotopy category $\mathrm{Ho}(\mathcal{M})$ from a simplicial model category $\mathcal{M}$, one cannot simply take the path components $\pi_0$ of the mapping spaces between arbitrary objects. 

Instead, one must restrict attention to the full subcategory of \textbf{fibrant-cofibrant objects}, denoted $\mathcal{M}_{cf}$. It is only within this subcategory that the simplicial mapping spaces $\mathrm{Map}_{\mathcal{M}}(X, Y)$ are guaranteed to be Kan complexes representing the correct derived mapping spaces. The morphisms in the homotopy category are thus given by:
\[
[X, Y]_{\mathrm{Ho}(\mathcal{M})} \cong \pi_0 \mathrm{Map}_{\mathcal{M}}(X, Y) \quad \text{for } X, Y \in \mathcal{M}_{cf}.
\]
For general objects $X, Y$, one must first replace them with weakly equivalent fibrant-cofibrant objects (via cofibrant replacement $QX$ and fibrant replacement $RY$) to compute this group.
\end{remark}
\endgroup}

{\begingroup
\newcommand{\C}{\mathcal{C}}
\newcommand{\D}{\mathcal{D}}
\newcommand{\Map}{\mathrm{Map}}
\newcommand{\Sp}{\text{Sp}}
\newcommand{\Exc}{\text{Exc}}
\newcommand{\Sfin}{\mathcal{S}_*^{\text{fin}}}
\newcommand{\uMap}{\underline{\mathrm{Map}}}
\newcommand{\Sone}{\mathbb{S}}
\newcommand{\Vlim}{\varprojlim}
\newcommand{\Clim}{\varinjlim}
\newcommand{\Ssphere}{\mathbb{S}}
\newcommand{\wedgeS}{\wedge}
\newcommand{\Stab}{\mathrm{Stab}} % 稳定化算子
\newcommand{\Sspace}{\mathcal{S}} % 空间范畴
\newcommand{\otimesD}{\otimes_{\mathcal{D}}}
\newcommand{\Unit}{\mathbf{1}_{\mathcal{D}}}

\newcommand{\SigInf}{\Sigma^\infty}    % Suspension Spectrum Functor
\newcommand{\OmeInf}{\Omega^\infty}    % Infinite Loop Space Functor


\subsection{Stable $\infty$-Category}

\begin{definition}[Loop Object]
For any object $X \in \C$, the loop object $\Omega X$ is the limit of the diagram $0 \to X \leftarrow 0$. It fits into the following pullback square:
\[
\begin{tikzcd}
\Omega X \arrow[r] \arrow[d] \arrow[dr, phantom, "\lrcorner", very near start] & 0 \arrow[d] \\
0 \arrow[r] & X
\end{tikzcd}
\]
Intuitively, $\Omega X \simeq 0 \times_X 0$.
\end{definition}

\begin{definition}[Suspension Object]
For any object $X \in \C$, the suspension object $\Sigma X$ is the colimit of the diagram $0 \leftarrow X \to 0$. It fits into the following pushout square:
\[
\begin{tikzcd}
X \arrow[r] \arrow[d] \arrow[dr, phantom, "\ulcorner", very near end] & 0 \arrow[d] \\
0 \arrow[r] & \Sigma X
\end{tikzcd}
§\]
Intuitively, $\Sigma X \simeq 0 \amalg_X 0$.
\end{definition}
\begin{definition}[Stable $\infty$-Category]
An $\infty$-category $\C$ is called \textbf{stable} if it satisfies the following conditions:
\begin{enumerate}
    \item There exists a zero object $0 \in \C$ (i.e., $\C$ is pointed).
    \item Every morphism in $\C$ admits a kernel and a cokernel.
    \item A triangle in $\C$ is a pushout square if and only if it is a pullback square.
\end{enumerate}
\end{definition}
\begin{definition}[The Stabilization of an $\infty$-Category]
Let $\mathcal{C}$ be an $\infty$-category admitting finite limits. The process of constructing the stable $\infty$-category associated to $\mathcal{C}$ proceeds in two stages:

\begin{enumerate}
    \item \textbf{Pointed View (Formation of $\mathcal{C}_*$):} \\
    First, we construct the \textit{pointed} $\infty$-category $\mathcal{C}_*$. Assuming $\mathcal{C}$ has a terminal object $*$, $\mathcal{C}_*$ is defined as the under-category of the terminal object:
    \[
    \mathcal{C}_* := \mathcal{C}_{*/} \cong \text{Fun}(\Delta^1, \mathcal{C}) \times_{\text{Fun}(\{0\}, \mathcal{C})} \{*\}
    \]
    Objects in $\mathcal{C}_*$ are morphisms $* \to X$ in $\mathcal{C}$ (i.e., objects equipped with a base point). In $\mathcal{C}_*$, the object corresponding to the identity $* \to *$ serves as a \textit{zero object} (both initial and terminal). Consequently, the loop functor $\Omega: \mathcal{C}_* \to \mathcal{C}_*$ is well-defined by $\Omega X = * \times_X *$.

    \item \textbf{Stabilization (Formation of $\text{Sp}(\mathcal{C})$):} \\
    The \textit{stabilization} of $\mathcal{C}$, denoted as $\text{Sp}(\mathcal{C})$ (or $\text{Stab}(\mathcal{C})$), is defined as the $\infty$-category of spectrum objects in $\mathcal{C}_*$. It is constructed as the homotopy limit of the tower of loop functors:
    \[
    \text{Sp}(\mathcal{C}) := \varprojlim \left( \cdots \xrightarrow{\Omega} \mathcal{C}_* \xrightarrow{\Omega} \mathcal{C}_* \xrightarrow{\Omega} \mathcal{C}_* \right)
    \]
    Explicitly, an object $E \in \text{Sp}(\mathcal{C})$ consists of a sequence $\{E_n\}_{n \geq 0}$ of objects in $\mathcal{C}_*$ together with equivalences $E_n \xrightarrow{\sim} \Omega E_{n+1}$ for each $n$. This construction forces the suspension functor $\Sigma$ to be an equivalence, rendering $\text{Sp}(\mathcal{C})$ a stable $\infty$-category.
\end{enumerate}
\end{definition}

\begin{theorem}[Universal Property of Stabilization]
Let $\C$ be an $\infty$-category with finite limits and a terminal object $\ast$. Let $\C_* = \C_{\ast/}$ be its pointed version. The stabilization $\Sp(\C)$ is characterized by the following equivalent descriptions:

\begin{enumerate}
    \item \textbf{Internal Construction (Loop Towers):} 
    $\Sp(\C)$ is the homotopy limit of the sequence of loop functors:
    \[
    \Sp(\C) \simeq \varprojlim \left( \cdots \xrightarrow{\Omega} \C_* \xrightarrow{\Omega} \C_* \xrightarrow{\Omega} \C_* \right)
    \]
    An object in $\Sp(\C)$ is an $\Omega$-spectrum, i.e., a sequence $\{E_n\}_{n \geq 0}$ in $\C_*$ with equivalences $E_n \simeq \Omega E_{n+1}$.

    \item \textbf{External Construction (Excision):} 
    $\Sp(\C)$ is equivalent to the $\infty$-category of pointed excisive functors from the category of finite pointed spaces $\Sfin$ to $\C$:
    \[
    \Sp(\C) \simeq \Exc_*(\Sfin, \C)
    \]
    A functor $F: \Sfin \to \C$ belongs to $\Exc_*(\Sfin, \C)$ if $F(\ast) \simeq \ast$ and $F$ maps every pushout square in $\Sfin$ to a pullback square in $\C$.
\end{enumerate}

Furthermore, the stabilization $\Sp(\C)$ is the universal stable $\infty$-category under $\C$: for any stable $\infty$-category $\mathcal{D}$, the functor $\Sp(\C) \to \mathcal{D}$ induces an equivalence of $\infty$-categories $\text{Fun}^{\text{lex}}(\Sp(\C), \mathcal{D}) \simeq \text{Fun}^{\text{lex}}(\C, \mathcal{D})$, where $\text{Fun}^{\text{lex}}$ denotes the $\infty$-category of left exact functors.
\end{theorem}

\begin{remark}[Distinction between $\Stab(\C)$ and $\Sp$]
It is essential to distinguish between the abstract process of stabilization and the specific category of spectra:
\begin{enumerate}
    \item \textbf{The Category of Spectra ($\Sp$):} Historically and by convention, $\Sp$ refers specifically to the stabilization of the $\infty$-category of pointed spaces $\Sspace_{\ast}$. That is:
    \[ \Sp \simeq \Stab(\Sspace) \simeq \varprojlim (\dots \xrightarrow{\Omega} \Sspace_{\ast} \xrightarrow{\Omega} \Sspace_{\ast}) \]
    This category serves as the unit object in the $\infty$-category of stable $\infty$-categories and provides the ground for stable homotopy theory.

    \item \textbf{Stabilization of an Arbitrary $\infty$-Category ($\Stab(\C)$):} For any $\infty$-category $\C$ with finite limits, $\Stab(\C)$ is the stable $\infty$-category constructed as the limit of the tower of loop functors:
    \[ \Stab(\C) = \varprojlim (\dots \xrightarrow{\Omega} \C_{\ast} \xrightarrow{\Omega} \C_{\ast}) \]
    While $\Stab(\C)$ is always a stable $\infty$-category, it may not possess a symmetric monoidal structure (like the smash product) unless $\C$ itself is equipped with a compatible monoidal structure.

    \item \textbf{The Relation:} Any stable $\infty$-category $\mathcal{D}$ is naturally tensored over $\Sp$. In this sense, $\Sp$ plays a role analogous to the ring of integers $\mathbb{Z}$ in abelian groups: for any $D \in \mathcal{D}$ and $E \in \Sp$, there is a well-defined object $E \otimes D \in \mathcal{D}$.
\end{enumerate}
\end{remark}

\begin{definition}[Internal Mapping Spectrum in $\Stab(\C)$]
Let $\C$ be a \textbf{closed symmetric monoidal $\infty$-category} $(\C, \otimes, \mathbf{1})$ that admits finite limits. Assume further that the tensor product $\otimes$ is compatible with the stabilization (i.e., it preserves colimits in each variable). 

Let $\mathcal{D} = \Stab(\C)$ be the resulting stable $\infty$-category, equipped with the induced symmetric monoidal structure $\otimesD$. For any objects $X, Y \in \mathcal{D}$, the \textbf{internal mapping spectrum} is defined as the object $\uMap_{\mathcal{D}}(X, Y) \in \mathcal{D}$ that satisfies the following conditions:

\begin{enumerate}
    \item \textbf{Adjunction Property:} It is the right adjoint to the tensor product functor. For any object $Z \in \mathcal{D}$, there is a natural equivalence of mapping spaces:
    \[ \Map_{\mathcal{D}}(Z \otimesD X, Y) \simeq \Map_{\mathcal{D}}(Z, \uMap_{\mathcal{D}}(X, Y)) \]
    
    \item \textbf{Spectrum Level Structure:} In terms of the sequence of objects $\{E_n\} \in \C_{\ast}$ representing the spectrum, the $n$-th level of the internal mapping spectrum is given by:
    \[ \uMap_{\mathcal{D}}(X, Y)_n \simeq \uMap_{\C}(X, \Sigma^n Y) \]
    where $\uMap_{\C}$ denotes the internal Hom in the underlying category $\C$ (if it exists) or the corresponding enrichment.
\end{enumerate}
\end{definition}

\begin{definition}[Mapping Spectrum]
Let $X$ and $Y$ be spectra in $\Sp$. The \textbf{mapping spectrum} from $X$ to $Y$, denoted as $\uMap(X, Y) \in \Sp$, is the unique spectrum (up to equivalence) characterized by the following properties:

\begin{enumerate}
    \item \textbf{Adjunction (Internal Hom):}
    For any spectrum $Z$, there is a natural equivalence of mapping spaces:
    \[
    \Map_{\Sp}(Z \wedgeS X, Y) \simeq \Map_{\Sp}(Z, \uMap(X, Y))
    \]
    This identifies $\uMap(X, Y)$ as the right adjoint to the smash product functor $( - \wedgeS X )$.

    \item \textbf{Omega-Spectrum Structure:}
    The $n$-th space of the mapping spectrum is equivalent to the space of maps from $X$ to the $n$-th suspension of $Y$:
    \[
    \uMap(X, Y)_n \simeq \Map_{\Sp}(X, \Sigma^n Y)
    \]
    The structure maps $\Sigma \uMap(X, Y)_n \to \uMap(X, Y)_{n+1}$ are induced by the stability of $\Sp$.
\end{enumerate}
\end{definition}


\begin{definition}[Homotopy Groups in $\Stab(\C)$]
Let $\C$ be a closed symmetric monoidal $\infty$-category with finite limits, and let $\D = \Stab(\C)$ be its stabilization with unit object $\Unit$. 
\begin{enumerate}
    \item \textbf{Homotopy Groups of an Object:} For any object $E \in \D$ and $n \in \mathbb{Z}$, the $n$-th homotopy group of $E$ is defined as the abelian group of homotopy classes of maps from the $n$-shifted unit object:
    \[ \pi_n(E) := [\Sigma^n \Unit, E]_{\D}  \]
    
    \item \textbf{Homotopy Groups of the Mapping Spectrum:} Let $\uMap_{\D}(X, Y) \in \D$ be the internal mapping spectrum between $X, Y \in \D$. Its homotopy groups characterize the graded morphisms between the two objects:
    \[ \pi_n \uMap_{\D}(X, Y) \cong [X, \Sigma^n Y]_{\D} \cong [\Sigma^{-n} X, Y]_{\D} \]
    where $[-, -]_{\D}$ denotes the set of homotopy classes i.e. the $0$-th homotopy group of the kan complex $\Map_{\D}(-, -)$.
\end{enumerate}
\end{definition}

\begin{remark}
The distinction lies in the target category:
\begin{itemize}
    \item $\Map_{\D}(X, Y) \in \mathcal{S}$ is a \textbf{space} (Kan complex). It represents the mapping space in the $\infty$-categorical sense.
    \item $\uMap_{\D}(X, Y) \in \D$ is a \textbf{spectrum} (Internal Hom). It is an object of the stable category $\D$ that stabilizes the mapping space.
\end{itemize}
In terms of homotopy groups: $\pi_n \Map_{\D}(X, Y) \cong \pi_n \uMap_{\D}(X, Y)$ for $n \ge 0$, because every spectrum is $\Omega$-spectrum in $\Stab(\C)$.
\end{remark}

\begin{construction}[Stabilization of a Suspension Spectrum]
Let $X \in \Sspace_{\ast}$ be a pointed space (or generally an object in a pointed $\infty$-category $\mathcal{C}$ with finite colimits). The construction of its associated \textbf{suspension spectrum} $\SigInf X \in \Sp$ proceeds as follows:

\begin{enumerate}
    \item \textbf{The Prespectrum Construction:}
    First, we form a \textit{prespectrum} $P_X$ by iterating the suspension functor $\Sigma$ on $X$. This is given by the sequence of spaces:
    \[ (P_X)_n := \Sigma^n X, \quad \text{for } n \ge 0 \]
    together with the structural maps (identities):
    \[ \sigma_n: \Sigma((P_X)_n) = \Sigma(\Sigma^n X) \xrightarrow{id} \Sigma^{n+1} X = (P_X)_{n+1} \]

    \item \textbf{Spectrification (The L-functor):}
    Since $P_X$ is not necessarily an $\Omega$-spectrum (i.e., the adjoint maps $(P_X)_n \to \Omega (P_X)_{n+1}$ are not equivalences), we apply the \textit{spectrification functor} $L$ (or stabilization) to convert it into a true spectrum. The resulting object is the suspension spectrum:
    \[ \SigInf X := L(P_X) \]
    Conceptually, the $k$-th space of this stable object is the colimit:
    \[ (\SigInf X)_k \simeq \operatorname*{colim}_{m \to \infty} \Omega^m \Sigma^{m+k} X \]

    \item \textbf{Universal Property (Adjunction):}
    The construction defines the left adjoint functor $\SigInf$ in the stabilization adjunction:
    \[
    \begin{tikzcd}
        \Sspace_{\ast}\arrow[r, "\SigInf", bend left] & \Sp \arrow[l, "\OmeInf", bend left]
    \end{tikzcd}
    \]
    where for any spectrum $E$, the right adjoint is given by $\OmeInf E := E_0$ (the 0-th space of the $\Omega$-spectrum $E$).
\end{enumerate}
\end{construction}

\begin{remark}[Bousfield-Friedlander Structure and Stabilization]
    The Bousfield-Friedlander model structure $\mathcal{M}_{\mathrm{BF}}$ on the category of prespectra is the left Bousfield localization of the strict model structure $\mathcal{M}_{\mathrm{strict}}$. The three classes of morphisms in $\mathcal{M}_{\mathrm{BF}}$ are characterized as follows:
    \begin{itemize}
        \item \textbf{Cofibrations:} These are exactly the same as the strict cofibrations (levelwise inclusions that satisfy the appropriate cell complex conditions).
        \item \textbf{Weak Equivalences:} These are the \textit{stable weak equivalences}, i.e., maps $f: X \to Y$ that induce isomorphisms on stable homotopy groups $\pi_n^{S}(X) \cong \pi_n^{S}(Y)$ for all $n \in \mathbb{Z}$.
        \item \textbf{Fibrations:} These are the maps that satisfy the right lifting property with respect to acyclic cofibrations. Specifically, a map $p: E \to B$ is a BF-fibration if it is a levelwise fibration and the square 
        \[
        \begin{tikzcd}
            E_n \arrow[r] \arrow[d, "p_n"] & \Omega E_{n+1} \arrow[d, "\Omega p_{n+1}"] \\
            B_n \arrow[r] & \Omega B_{n+1}
        \end{tikzcd}
        \]
        is a homotopy pullback for all $n$.
    \end{itemize}
    
    The transition from $\mathcal{M}_{\mathrm{strict}}$ to $\mathcal{M}_{\mathrm{BF}}$ captures the essence of stabilization. Since the fibrant objects in this structure are exactly the $\Omega$-spectra, the \textbf{fibrant replacement} of a prespectrum $X$ in $\mathcal{M}_{\mathrm{BF}}$ is precisely its \textbf{stabilization} (spectrification). 
    
    If $R_{\mathrm{BF}}$ denotes the fibrant replacement functor, we have a natural stable equivalence $j: X \xrightarrow{\sim} R_{\mathrm{BF}}(X)$, where $R_{\mathrm{BF}}(X)$ is an $\Omega$-spectrum. In the stable homotopy category, this is equivalent to the classical stabilization $QX = \Omega^\infty \Sigma^\infty X$:
    \[
    \begin{tikzcd}
        X \arrow[r, "j", "\sim_{\text{stable}}"'] \arrow[d, equal] & R_{\mathrm{BF}}(X) \arrow[d, "\simeq"] \\
        X \arrow[r, "\text{Stabilization}"] & QX
    \end{tikzcd}
    \]
    Thus, the Bousfield-Friedlander model structure provides the formal homotopy-theoretic machinery where "becoming an $\Omega$-spectrum" is equivalent to "becoming fibrant."
\end{remark}
\endgroup}
