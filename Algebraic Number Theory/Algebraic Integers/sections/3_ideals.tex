\section{Ideals}

\subsection{Dedekind Domains}

\textbf{Theorem(Properties of $\mathfrak{o}_K$).} The ring of integers $\mathfrak{o}_K$ in a number field $K$ is Noetherian, integrally closed, and every non-zero prime ideal is maximal.

\textbf{Definition} An integral domain $\mathfrak{o}$ is called a \textbf{Dedekind domain} if it satisfies the following conditions:
\begin{enumerate}
    \item It is \textbf{Noetherian}.
    \item It is \textbf{integrally closed}.
    \item Every non-zero prime ideal is \textbf{maximal}.
\end{enumerate}

\subsection{Factorization of Integral Ideals}

\textbf{Lemma} For every ideal $\mathfrak{a} \neq 0$ of a Dedekind domain $\mathfrak{o}$, there exist non-zero prime ideals $\mathfrak{p}_1, \dots, \mathfrak{p}_r$ such that:
\[
\mathfrak{a} \supseteq \mathfrak{p}_1 \mathfrak{p}_2 \cdots \mathfrak{p}_r.
\]

\textbf{Definition (Inverse of a Prime).} Let $\mathfrak{p}$ be a prime ideal. Define the set:
\[ \mathfrak{p}^{-1} = \{x \in K \mid x\mathfrak{p} \subseteq \mathfrak{o}\}. \]

\textbf{Lemma} Let $\mathfrak{p}$ be a prime ideal of $\mathfrak{o}$. For every ideal $\mathfrak{a} \neq 0$:
\[
\mathfrak{a}\mathfrak{p}^{-1} := \left\{ \sum a_i x_i \mid a_i \in \mathfrak{a}, x_i \in \mathfrak{p}^{-1} \right\} \neq \mathfrak{a}.
\]
Specifically, $\mathfrak{o} \subsetneq \mathfrak{p}^{-1}$ and $\mathfrak{p}\mathfrak{p}^{-1} = \mathfrak{o}$.

\textbf{Theorem(Unique Prime Factorization).} Every ideal $\mathfrak{a}$ of $\mathfrak{o}$ different from $(0)$ and $(1)$ admits a factorization
\[
\mathfrak{a} = \mathfrak{p}_1 \cdots \mathfrak{p}_r
\]
into non-zero prime ideals $\mathfrak{p}_i$ of $\mathfrak{o}$, which is \textbf{unique} up to the order of the factors.

\subsection{Fractional Ideals and the Ideal Group}

\textbf{Definition} A \textbf{fractional ideal} of $K$ is a finitely generated $\mathfrak{o}$-submodule $\mathfrak{a} \neq 0$ of $K$.

\textbf{Equivalent Definition:} An $\mathfrak{o}$-submodule $\mathfrak{a} \subset K$ ($\mathfrak{a} \neq 0$) is a fractional ideal if and only if there exists a non-zero element $c \in \mathfrak{o}$ such that $c\mathfrak{a} \subseteq \mathfrak{o}$ (i.e., $c\mathfrak{a}$ is an integral ideal).

\textbf{Proposition(Ideal Group).} The fractional ideals form an abelian group $J_K$, called the \textbf{ideal group} of $K$.
\begin{itemize}
    \item \textbf{Identity:} $(1) = \mathfrak{o}$.
    \item \textbf{Inverse:} The inverse of $\mathfrak{a}$ is:
    \[
    \mathfrak{a}^{-1} = \{x \in K \mid x\mathfrak{a} \subseteq \mathfrak{o}\}.
    \]
\end{itemize}

\textbf{Corollary} Every fractional ideal $\mathfrak{a}$ admits a unique representation as a product:
\[
\mathfrak{a} = \prod_{\mathfrak{p}} \mathfrak{p}^{\nu_{\mathfrak{p}}}
\]
with $\nu_{\mathfrak{p}} \in \mathbb{Z}$ and $\nu_{\mathfrak{p}} = 0$ for almost all $\mathfrak{p}$.
Thus, $J_K$ is the \textbf{free abelian group} on the set of non-zero prime ideals $\mathfrak{p}$ of $\mathfrak{o}$.

\subsection{The Class Group}

\textbf{Principal Fractional Ideals ($P_K$).} The fractional ideals of the form $(a) = a\mathfrak{o}$ for $a \in K^*$ form a subgroup of $J_K$ denoted by $P_K$.

\textbf{Ideal Class Group ($Cl_K$).} The quotient group:
\[
Cl_K = J_K / P_K
\]
is called the \textbf{ideal class group} of $K$.

\textbf{Fundamental Exact Sequence}:
The relation between numbers and ideals is captured by the exact sequence:
\begin{equation*}
    \begin{tikzcd}
        1 \arrow[r] & \mathfrak{o}^* \arrow[r, hook] & K^* \arrow[r, "a \mapsto (a)"] & J_K \arrow[r, "proj"] & Cl_K \arrow[r] & 1
    \end{tikzcd}
\end{equation*}

\hrulefill