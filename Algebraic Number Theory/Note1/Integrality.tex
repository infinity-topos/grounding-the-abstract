\documentclass[10pt, a4paper, twocolumn]{article}
\usepackage[margin=0.5in]{geometry}
\usepackage{amsmath, amssymb, amsthm}
\usepackage{xcolor}
\usepackage{enumitem}
\usepackage{titlesec}

% Compact spacing
\setlength{\parindent}{0pt}
\setlist{nosep}
\titlespacing*{\section}{0pt}{5pt}{2pt}
\titlespacing*{\subsection}{0pt}{3pt}{1pt}

\title{\vspace{-1cm}Algebraic Number Theory - Chapter I: Integers (Summary)}
\date{}
\author{}

\begin{document}
{\let\newpage\relax\maketitle}
\vspace{-1.5cm}

\section*{§ 1. The Gaussian Integers}

\textbf{Ring of Integers:} The ring $\mathbb{Z}[i] = \{a+bi \mid a,b \in \mathbb{Z}\}$ is the integral closure of $\mathbb{Z}$ in the field $\mathbb{Q}(i)$. It consists precisely of the elements in $\mathbb{Q}(i)$ satisfying a monic polynomial in $\mathbb{Z}[x]$.

\textbf{Prime Elements (up to units):}
\begin{enumerate}
    \item $\pi = 1+i$ (associated to 2).
    \item $\pi = a+bi$ with $a^2+b^2=p$, where $p \in \mathbb{Z}$ is a prime with $p \equiv 1 \pmod 4$.
    \item $\pi = p$, where $p \in \mathbb{Z}$ is a prime with $p \equiv 3 \pmod 4$.
\end{enumerate}

\hrulefill

\section*{§ 2. Integrality}

\subsection*{1. Definitions \& Transitivity}
\begin{itemize}
    \item \textbf{Algebraic Number Field:} Finite extension $K|\mathbb{Q}$.
    \item \textbf{Algebraic Integer:} Root of a monic $f(x) \in \mathbb{Z}[x]$.
    \item \textbf{Integral over $A$:} $b \in B$ is integral over $A \subseteq B$ if it satisfies $x^n + a_1x^{n-1} + \dots + a_n = 0$ ($a_i \in A$).
    \item \textbf{Integral Closure:} $\overline{A} = \{b \in B \mid b \text{ integral over } A\}$.
    \item \textbf{Normalization:} The integral closure of a domain $A$ in its fraction field.
    \item \textbf{Transitivity (Prop 2.4):} If $A \subseteq B \subseteq C$ are rings, $B$ integral over $A$, $C$ integral over $B \implies C$ integral over $A$.
\end{itemize}

\subsection*{2. Row-Column Expansion}
Let $M$ be an $r \times r$ matrix, $M^*$ its adjoint. Then $MM^* = \det(M)I$. Implication:
\[ Mx = 0 \implies \det(M)x = 0. \]
(Used to prove: finite generation $\iff$ integrality).

\subsection*{3. Integrality in Extensions}
Let \textcolor{red}{$A$ be an integrally closed domain}, $K = \text{frac}(A)$, $L|K$ a finite extension, and $B$ the integral closure of $A$ in $L$.
\begin{itemize}
    \item Every $\beta \in L$ can be written as $\beta = b/a$ with $b \in B, a \in A$.
    \item An element $\beta \in L$ belongs to $B$ if and only if its minimal polynomial over $K$ lies in $A[x]$.
\end{itemize}

\subsection*{4. Trace, Norm, Characteristic Poly.}
For $x \in L$, let $T_x: L \to L$ be the map $\alpha \mapsto x\alpha$.
\begin{itemize}
    \item \textbf{Char. Poly:} $f_x(t) = \det(tI - T_x)$.
    \item \textbf{Trace:} $Tr_{L/K}(x) = \text{trace}(T_x)$.
    \item \textbf{Norm:} $N_{L/K}(x) = \det(T_x)$.
\end{itemize}
If $L|K$ is separable with embeddings $\sigma: L \to \bar{K}$:
\[ f_x(t) = \prod_\sigma (t - \sigma x), \quad Tr(x) = \sum_\sigma \sigma x, \quad N(x) = \prod_\sigma \sigma x. \]
\textbf{Tower Property:} $K \subseteq L \subseteq M$.
\[ Tr_{M/K} = Tr_{L/K} \circ Tr_{M/L}, \quad N_{M/K} = N_{L/K} \circ N_{M/L}. \]

\subsection*{5. Discriminant Calculations}
For a basis $\alpha_1, \dots, \alpha_n$ of $L|K$ (separable):
\[ d(\alpha_1, \dots, \alpha_n) = \det(\sigma_i \alpha_j)^2 = \det(Tr_{L/K}(\alpha_i \alpha_j)). \]
If the basis is $1, \theta, \dots, \theta^{n-1}$ (power basis):
\[ d(1, \theta, \dots, \theta^{n-1}) = \prod_{i < j} (\theta_i - \theta_j)^2 \quad (\text{Vandermonde}). \]

\subsection*{6. Bilinear Form}
The trace defines a bilinear form $L \times L \to K$:
\[ (x, y) \mapsto Tr_{L/K}(xy). \]
It is \textbf{non-degenerate} if $L|K$ is separable (i.e., discriminant $\neq 0$). With basis $\{\alpha_i\}$, it corresponds to matrix $M_{ij} = Tr(\alpha_i \alpha_j)$.

\subsection*{7. Integrality of Trace and Norm}
If $A$ is integrally closed and $x \in B$ (integral closure), then:
\[ Tr_{L/K}(x) \in A \quad \text{and} \quad N_{L/K}(x) \in A. \]
\textbf{Units and Norms:} Since the norm is multiplicative, an element $x \in B$ is a unit if and only if its norm is a unit in $A$:
\[ x \in B^\times \iff N_{L/K}(x) \in A^\times. \]
(e.g., for $A=\mathbb{Z}$, $x$ is a unit $\iff N(x) = \pm 1$).

\subsection*{8. Localization of the Discriminant}
Let $\alpha_1, \dots, \alpha_n \in B$ be a basis of $L|K$ with discriminant $d$. Then:
\[ \textcolor{red}{d B \subseteq A\alpha_1 + \dots + A\alpha_n}. \]

\subsection*{9. Integral Basis}
An integral basis is a basis $\omega_1, \dots, \omega_n$ of $L|K$ such that $B = A\omega_1 + \dots + A\omega_n$.
\begin{itemize}
    \item \textcolor{red}{If $A$ is a PID}, then every finitely generated submodule $M \neq 0$ of $L$ is a free $A$-module of rank $[L:K]$. Thus, $B$ admits an integral basis.
\end{itemize}

\subsection*{10. Discriminant of Algebraic Integers}
Let $\mathcal{o}_K$ be the ring of integers of $K$. The discriminant of $K$ (or an ideal $\mathfrak{a}$) is defined via a $\mathbb{Z}$-basis $\alpha_1, \dots, \alpha_n$ of $\mathcal{o}_K$ (or $\mathfrak{a}$):
\[ d(\mathfrak{a}) = d(\alpha_1, \dots, \alpha_n). \]
\begin{itemize}
    \item It is independent of the choice of basis.
    \item $d(\mathfrak{a}) \neq 0$ implies linear independence.
    \item Relation: If $\mathfrak{a} \subseteq \mathfrak{a}'$, then $d(\mathfrak{a}) = (\mathfrak{a}' : \mathfrak{a})^2 d(\mathfrak{a}')$.
\end{itemize}

\end{document}